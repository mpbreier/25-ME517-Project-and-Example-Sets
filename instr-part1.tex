\section*{Project I: Topic ID and Overview (due \textcolor{red}{Sept 19})}

%This is just a placeholder for now

The first step in your semester-long research proposal development is to select a topic area in the mechanics of soft materials that's sufficiently interesting to you. 
It may be helpful to think of the proposal-writing process as the following. 
You want to study something that you are especially interested in, but you don't yet have the resources that you need to pursue this fully. 
Your job is (eventually) to communicate what it is you want to study, why it's worthwhile to be studied, and enumerate all of the reasons it's in some benefactor's interest to provide you the support that you need.
The particular benefactor we will leverage is the National Science Foundation, which cares about making fundamental ``vertical'' advances in fields (known as their \textit{Intellectual Merit} criterion) and having their funded projects improve society (known as the \textit{Broader Impact} to society criterion). 

Aim for approximately 500 words of total text, such as to reflect the important three Cs: \textbf{\textit{clear}}, \textbf{\textit{concise}}, and \textbf{\textit{compelling}}.
Your submission should be structured in three sections as separated below, and should address the following points: 

\begin{enumerate}
\item \textbf{Statement of Research Interest (why you personally want to study the subject)}
\begin{itemize}
\item Describe an area or phenomenon in the mechanics of soft materials that you find compelling. 
\item What motivates your interest and pursuit of this subject (e.g., your current or developing expertise, research interests, or otherwise)? \textit{Note: This may be more personal or anecdotal and is for my own understanding of your topic selection!}
\end{itemize}
\item \textbf{Intellectual Merit (why it is objectively worth delving deeper)}
\begin{itemize}
\item Describe, to someone with expertise in mechanics but perhaps not your system of interest, the core scientific principles underpinning (or perhaps, enabling development in) your topic of interest. 
\item Given the course syllabus, how will particular material we will cover this semester relate to what you propose? What background information do you need to do not just a good, but great, job in proposing something interesting? 
\end{itemize}
\item \textbf{Broader Impact (who, or what, does studying this area benefit?)}
\begin{itemize}
\item How might advances you envision in this area be impactful beyond your own interest? 
\item What does a ``winning scenario'' in this area look like? Briefly describe who might benefit (e.g., particular industries, health/science sectors, the public) and how that could plausibly happen.
\end{itemize}
\end{enumerate}

\emph{A strong submission will clearly illustrate your personal goals with this project, and show how your interest could manifest as advances in the broader field and society.}

\newpage
\section*{Project Checkpoint I: Topic ID and Overview}

An area of soft material mechanics that is quite compelling is the subject of topological optimization. Specifically, it is intriguing to consider how this process can help minimize sample size, sample complexity, loading condition complexity, etc. while maximizing material property information (such as a bulk modulus). Depending on what one is seeking specifically, sample geometry will vary significantly. It is quite interesting to me just how varied and unique optimized samples can potentially be. Through this optimization, one might discover optimized samples that seem odd or atypical, with shapes that someone might not consider creating otherwise.

Along with this intrigue, my lab currently has a project that involves the fine tuning of an open source topological optimization code. Pairing this with a full field method such as using magnetic resonance (MR) to determine the full three-dimensional displacement field of a soft material sample poses the framework of a fascinating project to verify this topological optimization code, with potentially very useful results.

Broadly, topological optimization is a method of creating the most "efficient" sample design. This process is typically driven through iterative simulation. How efficient is defined depends on one's specific desire out of a sample. When working with expensive materials, someone might look at how to minimize sample volume while maximizing strength. In researching soft materials, testing samples can become quite complicated. They can be hard to grip, exhibit nonlinear deformation, and exhibit material properties many orders of magnitude smaller than harder materials like metals, which may make testing by more standardized means more difficult. Thus, one may seek to optimize sample design to yield the most information on a sample's behavior based on specified testing constraints.

In order to do this, one needs to understand the way soft materials deform. To verify a working topological optimization code that is specifically aimed at optimizing soft materials. one needs to know what potential soft material responses will look like, and how to write them up in a numerical way (e.g. via a finite element method). Specifically, the hyperelastic portion of the course s=will be potentially very useful, as hyperelastic materials such as silicone are what I currently plan to use in validating this TopOpt code via experimentation.

There exist many circumstances in which one might require a way to measure material response despite an inability to do so, whether that stem from an inability to do so non-invasively or a lack of ability to feasibly test a material in the required environment. Maybe the loading conditions placed on a sample are complex and such a setup does not allow for one to reliably and consistently measure strain response. In the biomedical field, tissues are soft in most cases and need to be observed with intense care. Thus, it would be considerably useful if there was a reliable, trusted method to create samples that model behavior created by complex stress conditions in a more easily measurable way.

A successful outcome of this project would look like a strong validation of the TopOpt results. This success would greatly benefit researchers in the field of soft material experimentation as this outcome would signify the reliability of a working sample optimization method through a code that is all open source.




%% 
%% Copyright 2007-2025 Elsevier Ltd
%% 
%% This file is part of the 'Elsarticle Bundle'.
%% ---------------------------------------------
%% 
%% It may be distributed under the conditions of the LaTeX Project Public
%% License, either version 1.3 of this license or (at your option) any
%% later version.  The latest version of this license is in
%%    http://www.latex-project.org/lppl.txt
%% and version 1.3 or later is part of all distributions of LaTeX
%% version 1999/12/01 or later.
%% 
%% The list of all files belonging to the 'Elsarticle Bundle' is
%% given in the file `manifest.txt'.
%% 
%% Template article for Elsevier's document class `elsarticle'
%% with harvard style bibliographic references

\documentclass[preprint,12pt,authoryear]{elsarticle}

\usepackage[utf8]{inputenc}
\usepackage[margin=2cm]{geometry}
\usepackage{graphicx}
\usepackage{multirow}
\usepackage{amssymb}
\usepackage{amsmath}
\usepackage{setspace}
\usepackage{outlines}
\usepackage{enumitem}
\usepackage{xcolor}
\usepackage{upgreek}
\usepackage{mathabx}
\usepackage{algorithm}
\usepackage{algorithmic}
\usepackage{amsthm}
\usepackage[labelfont=bf,font=small]{caption}
\usepackage{epsfig}
\usepackage{geometry}
\usepackage{subfigure}
\usepackage[textsize=tiny]{todonotes}
\usepackage[normalem]{ulem}
\usepackage{lipsum}
\usepackage{array}
\usepackage{booktabs}
\usepackage{lineno}
\usepackage{array}
\usepackage{tikz}
\usepackage[english]{babel}
\usepackage{animate}

\usepackage[]{siunitx}
\sisetup{range-units=single,separate-uncertainty = true,print-unity-mantissa=false,per-mode=symbol,range-phrase = \text{--},
inter-unit-product=\cdot
}

\usepackage[
    protrusion=true,
    activate={true,nocompatibility},
    final,
    tracking=true,
    kerning=true,
    spacing=true,
    factor=1100]{microtype}
    
\SetTracking{encoding={*}, shape=sc}{40}

\usepackage{outlines}
\usepackage{enumitem}

\definecolor{lightblue}{rgb}{0.63, 0.74, 0.78}
\definecolor{seagreen}{rgb}{0.18, 0.42, 0.41}
\definecolor{orange}{rgb}{0.85, 0.55, 0.13}
\definecolor{silver}{rgb}{0.69, 0.67, 0.66}
\definecolor{rust}{rgb}{0.72, 0.26, 0.06}
\definecolor{purp}{RGB}{68, 14, 156}

\definecolor{zblue}{RGB}{8,81,156}
\definecolor{zpurp}{RGB}{84,39,143}
\definecolor{zred}{RGB}{165,15,21}

\colorlet{lightrust}{rust!50!white}
\colorlet{lightorange}{orange!25!white}
\colorlet{lightlightblue}{lightblue}
\colorlet{lightsilver}{silver!30!white}
\colorlet{darkorange}{orange!75!black}
\colorlet{darksilver}{silver!65!black}
\colorlet{darklightblue}{lightblue!65!black}
\colorlet{darkrust}{rust!85!black}
\colorlet{darkseagreen}{seagreen!85!black}



\usepackage{hyperref}
\hypersetup{
  colorlinks=true,
}
\usepackage{tabularx}
\usepackage{bbm}
\usepackage{bm}

\usepackage[nameinlink]{cleveref}
\crefname{equation}{}{}
\def\appendixname{}
\crefname{appendix}{}{}


\usepackage{setspace}
% \doublespacing
\setlength{\heavyrulewidth}{1.5pt}
% \setlength{\abovetopsep}{4pt}
 
\usepackage{soul}
\sethlcolor{yellow}

\usepackage[parfill]{parskip}

\usepackage{lineno}
\usepackage{tcolorbox}
%\linenumbers

%% Self-defined Macros
\newcommand{\bn}[1]{\mathbf{#1}}
\newcommand{\bs}[1]{\ensuremath{\boldsymbol{#1}}}
\newcommand{\mc}[1]{\ensuremath{\mathcal{#1}}}
\newcommand{\norm}[1]{\ensuremath \lVert #1 \rVert}
\newcommand{\GD}[1]{ D #1 (\boldsymbol{F})[\boldsymbol{H}]}
\newcommand{\GDI}[1]{ D #1 (\boldsymbol{I})[\boldsymbol{H}]}
\newcommand{\Lin}[1]{ L_{\boldsymbol{F}} #1 [\boldsymbol{H}]}
\newcommand{\LinI}[1]{ L_{\boldsymbol{I}} #1 [\boldsymbol{H}]}
\newcommand{\Dm}[1]{\ensuremath{\boldsymbol{\varphi} } }
\newcommand{\tr}[1]{\textcolor{red}{#1}}
\newcommand{\ignore}[1]{}
\newcommand{\fracp}[2]{\frac{\partial #1}{\partial #2}}
\newcommand{\wt}[1]{\widetilde{#1}}
\renewcommand{\[}{\left[}
\renewcommand{\]}{\right]}
\renewcommand{\(}{\left(}
\renewcommand{\)}{\right)}
\newcommand{\tn}{\textnormal}
\newcommand{\gradX}{\nabla_{\bm{X}}}
\newcommand{\gradx}{\nabla_{\bm{x}}}
\newcommand{\gradF}{\nabla_{\bn{F}}}


%% Coloring for comments
\usepackage{color,tikz,soul}
\newcommand{\jbe}[1]{\textcolor{violet}{#1}}

\newcommand{\rev}[1]{\textcolor{red}{#1}}

\usepackage{xcolor, soul}
\sethlcolor{yellow}
\usepackage[per-mode=symbol]{siunitx}

\setlength{\tabcolsep}{12pt}

\renewcommand{\thetable}{\arabic{table}} % Originally Roman uppercase

%% Use the option review to obtain double line spacing
%% \documentclass[authoryear,preprint,review,12pt]{elsarticle}

%% Use the options 1p,twocolumn; 3p; 3p,twocolumn; 5p; or 5p,twocolumn
%% for a journal layout:
%% \documentclass[final,1p,times,authoryear]{elsarticle}
%% \documentclass[final,1p,times,twocolumn,authoryear]{elsarticle}
%% \documentclass[final,3p,times,authoryear]{elsarticle}
%% \documentclass[final,3p,times,twocolumn,authoryear]{elsarticle}
%% \documentclass[final,5p,times,authoryear]{elsarticle}
%% \documentclass[final,5p,times,twocolumn,authoryear]{elsarticle}

%% For including figures, graphicx.sty has been loaded in
%% elsarticle.cls. If you prefer to use the old commands
%% please give \usepackage{epsfig}

%% The amssymb package provides various useful mathematical symbols
%\usepackage{amssymb}
%% The amsmath package provides various useful equation environments.
%\usepackage{amsmath}
%% The amsthm package provides extended theorem environments
%% \usepackage{amsthm}

%% The lineno packages adds line numbers. Start line numbering with
%% \begin{linenumbers}, end it with \end{linenumbers}. Or switch it on
%% for the whole article with \linenumbers.
%% \usepackage{lineno}

\journal{ME517---Mechanics of Soft Materials}

\setlength{\marginparwidth}{2cm}
\begin{document}

\begin{frontmatter}

\title{MECHENG 517---Mechanics of Soft Materials} %% Article title

%% Make a new file, e.g. "main.tex", and then replace my name with yours!
\author{Prof. Jon Estrada} 


\affiliation{organization={University of Michigan},%Department and Organization
            addressline={2350 Hayward St.}, 
            city={Ann Arbor},
            postcode={48105}, 
            state={MI},
            country={USA}}

%% Abstract
% \begin{abstract}
% %% Text of abstract
% Abstract text.
% \end{abstract}

%%Graphical abstract
% %\includegraphics{grabs}
% \begin{graphicalabstract}
% \end{graphicalabstract}

%%Research highlights
% \begin{highlights}
% \item Research highlight 1
% \item Research highlight 2
% \end{highlights}

%% Keywords
% \begin{keyword}
%% keywords here, in the form: keyword \sep keyword

%% PACS codes here, in the form: \PACS code \sep code

%% MSC codes here, in the form: \MSC code \sep code
%% or \MSC[2008] code \sep code (2000 is the default)

% \end{keyword}

\end{frontmatter}

%% If you want to include line numbers, uncomment the line below this:
% \linenumbers

\section*{Proposing a Research Topic in the Mechanics of Soft Materials}

As your major deliverable this semester, you are going to develop a ``white paper'' proposal for a research project in the area of soft material mechanics. 
This is deliberately a bit different from a standard report by focusing on being forward-looking. 
A white paper is a $3-5$ page document that contains:
\begin{enumerate}
    \item A short \textbf{overview} of your topic
    \item A brief summary of the \textbf{current}, or state-of-the-art of \textbf{knowledge} in that area
    \item The \textbf{knowledge gap} you are interested in filling
    \item The \textbf{long-term goal} of what your project would do in the field if successful
    \item The testable \textbf{central hypothesis} governing the work
    \item Three \textbf{specific aims} you would pursue with their own working hypotheses
    \item \textbf{Preliminary data} backing up why you formulated and believe those hypotheses
    \item The \textbf{expected outcomes} or products of your proposed research that link back to the specific aims  
\end{enumerate}

This project development will occur over the course of the entire semester, and every assignment you have this semester will have a component that will develop this project proposal in some way. 
The schedule will be as follows:
\smallskip

\footnotesize
\begin{tabularx}{\textwidth}{ccXX}

\textbf{Checkpoint} & \textbf{Due Date} & \textbf{Major Component(s)} & \textbf{White Paper Section(s)} \\
\hline
\hline
P1 & Sept 17 & Topic ID and overview & Overview, long-term goal \\
\hline
P2 & Oct 1 & Literature review & Current knowledge and gap \\
\hline
P3 & Oct 15 & Data gathering and summary & Preliminary data \\
\hline
P4 & Nov 5 & Data analysis and & Central hypothesis \\
 &  & central hypothesis & Support from preliminary data \\
\hline
P5 & Nov 19 & Three specific aims and & Specific aims \\
 & & supported working hypotheses & Expected outcomes \\
\hline
P6 & Dec 8 & Full white paper draft & All sections \\
 &  & (and peer review) &  \\
\hline
-- & Dec 15--16 & Oral presentations & Final draft \\
\hline
\end{tabularx}
\normalsize

\newpage


% Additional notes sections for ME517

%\include{instr-mathnotation}


% \include{instr-part6}
% \include{instr-part5}
% \include{instr-part4}
% \section{Data Gathering and Summary}

%This is just a placeholder for now

\section*{Project II: Literature Review (due Oct 3)}

The second step in your semester-long research proposal development is to contextualize your problem within the current field of your choice, demonstrate your understanding of the state-of-the-art, and identify something we don't yet understand but need to---this is the gap your (hypothetical) proposed project would seek to fill.

You'll execute this portion of the project as an outline. 
This outline does not need to be long! 
It does, however, need to be very clear, as you'll be expanding on it later. 
The sections should be the following:

\renewcommand{\outlinei}{enumerate}
\renewcommand{\outlineii}{itemize}
\begin{outline}
    \1 \textbf{Introductory context}
        \2 Add one or two bullet points briefly framing the history of your topic and its significance. 
        \2 Citations here are important---there should be several well-placed citations in each of these bullet points, and good review papers are especially helpful to lean on for context. 
        \2 \textit{\textbf{Example}: While gradient materials occur widely in nature---the structurally protective gradient of squid beaks [1], junctions between ligaments and bones [2], and the byssus threads that hold mussels to rocks [3] to name a few---polymeric gradient materials were only first considered in an engineering context in 1972 [4].}
    \1 \textbf{The state of the field}
        \2 In one or two bullet points, explain how things are done in the area of your project at the present moment. 
        \2 This can include any of experimentation, computational methods, and/or theory.
        \2 \textit{\textbf{Example 1}: Compositional gradients in engineered materials are typically produced in one of three ways: spatially (1) varying polymer crosslink density, e.g. using ultraviolet light-sensitive reactive groups [5, 6], (2) seeding of micro- or nanoparticles using centrifugation [7], electric fields [8], or other methods [9], or (3) using porosity to create structural gradients using dissolvable template-making materials [10].}
        \2 \textit{\textbf{Example 2}: Additive manufacturing has emerged as perhaps the best option for making complex functionally gradient soft materials on the order of cm or larger [11-14].}
    \1 \textbf{The Big Gap}
        \2 In one or two sentences, what is it that we don't know, and why isn't it solved? 
        \2 \textit{\textbf{Example 1:} The principal limitation of the first three techniques is scalability to larger sizes.}
        \2 \textit{\textbf{Example 2:} Arguably, the most challenging barrier for widespread production of gradient materials is the combination of sample repeatability and a comprehensive lack of validation options.} 
\end{outline}

\textit{A strong literature review outline will succinctly illustrate the essential context of the problem, the current best knowledge in the area, and the critical gap in knowledge restricting further advances or implementation, and should cite approximately 10-15 references with a \LaTeX ~bibliography.}

\newpage
\section*{Project Checkpoint II: Lit Review Outline}
\begin{outline}
    \1 \textbf{Introductory context}
        \2 When it comes to mechanically characterizing materials, standard testing methods (e.g. uniaxial tensile tests), while being relatively simple to run, only are able to obtain data at a few locations through sensors/gauges and rely on analytical stress distributions in simple geometries to find material properties [\cite{OlderRevMatlTesting}]. Although these tests are sufficient enough for materials such as metals, more complex tests that yield more data are required to reliably characterize materials with more complex constitutive behavior (e.g. hyperelastic or viscoelastic materials) [\cite{FunctionalMRI}, \cite{TheVirtualFieldsMethod}].
    \1 \textbf{The state of the field}
        \2 Full-field measurements, having become more robust and widespread in recent years, are actively replacing old standards for testing as they allow one to get more data in each test [\cite{MRuPaper1}, \cite{DICbook}]. This both decreases testing effort and increases material model accuracy.
        \2 Designing samples through topological optimization allows for one to achieve maximum richness in data from a minimum amount of testing [\cite{TopOptEntropyBased}] when paired with the data collection of more recent full-field techniques like DIC and MR-u [\cite{BriefRevMatlTesting}, \cite{nikolov2025variationmatchingsensitivitybasedvirtualfields}].
    \1 \textbf{The Big Gap}
        \2 One of the greatest problems in this growing area is choosing what quantities to which you should optimize your specimen geometries. Furthermore, the lack of ample experimental validation of optimized geometries makes it difficult to be certain how well these varied choices in optimized geometries will perform in practice.
        \2 Another issue present is the simultaneous need to bring down computation costs while ensuring the global convergence of optimized geometries, as mesh size should not play a factor in the output geometry. 
\end{outline}
%\section*{Project I: Topic ID and Overview (due \textcolor{red}{Sept 19})}

%This is just a placeholder for now

The first step in your semester-long research proposal development is to select a topic area in the mechanics of soft materials that's sufficiently interesting to you. 
It may be helpful to think of the proposal-writing process as the following. 
You want to study something that you are especially interested in, but you don't yet have the resources that you need to pursue this fully. 
Your job is (eventually) to communicate what it is you want to study, why it's worthwhile to be studied, and enumerate all of the reasons it's in some benefactor's interest to provide you the support that you need.
The particular benefactor we will leverage is the National Science Foundation, which cares about making fundamental ``vertical'' advances in fields (known as their \textit{Intellectual Merit} criterion) and having their funded projects improve society (known as the \textit{Broader Impact} to society criterion). 

Aim for approximately 500 words of total text, such as to reflect the important three Cs: \textbf{\textit{clear}}, \textbf{\textit{concise}}, and \textbf{\textit{compelling}}.
Your submission should be structured in three sections as separated below, and should address the following points: 

\begin{enumerate}
\item \textbf{Statement of Research Interest (why you personally want to study the subject)}
\begin{itemize}
\item Describe an area or phenomenon in the mechanics of soft materials that you find compelling. 
\item What motivates your interest and pursuit of this subject (e.g., your current or developing expertise, research interests, or otherwise)? \textit{Note: This may be more personal or anecdotal and is for my own understanding of your topic selection!}
\end{itemize}
\item \textbf{Intellectual Merit (why it is objectively worth delving deeper)}
\begin{itemize}
\item Describe, to someone with expertise in mechanics but perhaps not your system of interest, the core scientific principles underpinning (or perhaps, enabling development in) your topic of interest. 
\item Given the course syllabus, how will particular material we will cover this semester relate to what you propose? What background information do you need to do not just a good, but great, job in proposing something interesting? 
\end{itemize}
\item \textbf{Broader Impact (who, or what, does studying this area benefit?)}
\begin{itemize}
\item How might advances you envision in this area be impactful beyond your own interest? 
\item What does a ``winning scenario'' in this area look like? Briefly describe who might benefit (e.g., particular industries, health/science sectors, the public) and how that could plausibly happen.
\end{itemize}
\end{enumerate}

\emph{A strong submission will clearly illustrate your personal goals with this project, and show how your interest could manifest as advances in the broader field and society.}

\newpage
\section*{Project Checkpoint I: Topic ID and Overview}

An area of soft material mechanics that is quite compelling is the subject of topological optimization. Specifically, it is intriguing to consider how this process can help minimize sample size, sample complexity, loading condition complexity, etc. while maximizing material property information (such as a bulk modulus). Depending on what one is seeking specifically, sample geometry will vary significantly. It is quite interesting to me just how varied and unique optimized samples can potentially be. Through this optimization, one might discover optimized samples that seem odd or atypical, with shapes that someone might not consider creating otherwise.

Along with this intrigue, my lab currently has a project that involves the fine tuning of an open source topological optimization code. Pairing this with a full field method such as using magnetic resonance (MR) to determine the full three-dimensional displacement field of a soft material sample poses the framework of a fascinating project to verify this topological optimization code, with potentially very useful results.

Broadly, topological optimization is a method of creating the most "efficient" sample design. This process is typically driven through iterative simulation. How efficient is defined depends on one's specific desire out of a sample. When working with expensive materials, someone might look at how to minimize sample volume while maximizing strength. In researching soft materials, testing samples can become quite complicated. They can be hard to grip, exhibit nonlinear deformation, and exhibit material properties many orders of magnitude smaller than harder materials like metals, which may make testing by more standardized means more difficult. Thus, one may seek to optimize sample design to yield the most information on a sample's behavior based on specified testing constraints.

In order to do this, one needs to understand the way soft materials deform. To verify a working topological optimization code that is specifically aimed at optimizing soft materials. one needs to know what potential soft material responses will look like, and how to write them up in a numerical way (e.g. via a finite element method). Specifically, the hyperelastic portion of the course s=will be potentially very useful, as hyperelastic materials such as silicone are what I currently plan to use in validating this TopOpt code via experimentation.

There exist many circumstances in which one might require a way to measure material response despite an inability to do so, whether that stem from an inability to do so non-invasively or a lack of ability to feasibly test a material in the required environment. Maybe the loading conditions placed on a sample are complex and such a setup does not allow for one to reliably and consistently measure strain response. In the biomedical field, tissues are soft in most cases and need to be observed with intense care. Thus, it would be considerably useful if there was a reliable, trusted method to create samples that model behavior created by complex stress conditions in a more easily measurable way.

A successful outcome of this project would look like a strong validation of the TopOpt results. This success would greatly benefit researchers in the field of soft material experimentation as this outcome would signify the reliability of a working sample optimization method through a code that is all open source.





%% The appendix will contain the example problems assigned as problem sets. 
\appendix

%include{instr-PS5}
%\include{instr-PS4}
%\setcounter{section}{3} % This causes the next section to be Appendix B


\section*{Examples III. Linear Viscoelastic Models}
\label{PS3}

This set of example problems is due on October 17, 2025. 

% This is a placeholder for the example problems from the third problem set. 
% You'll replace this file with the one I supply on canvas. 

\medskip
\subsection*{3--1. \textbf{Converting creep to relaxation} [4 pts].} 
Say we measure the creep function for a material by fitting a sum of exponential functions to some data. 
We determine the creep function to be 
\begin{equation}
    J_c(t) = \frac{1}{1000}\left(10 - 5 e^{-t/4} - 3e^{-t/8}\right).
\end{equation}
(a) Attach a plot of $J_c(t)$, labeling significant values.

(b) Determine the corresponding stress relaxation function $G_r(t)$. What are the characteristic stress relaxation times now, and how do they compare to the creep relaxation times? 

\bigskip
\bigskip
\bigskip
\subsection*{3--2. \textbf{Alternate standard linear solid model} [4 pts].}

In class, we derived the relaxation and creep compliance functions $G_r(t)$ and $J_c(t)$ for a standard linear solid (SLS) model consisting of a spring in parallel with a Maxwell branch. 
In this question, we'll investigate a variant arrangement for the SLS, where a spring is placed in series with a Kelvin-Voigt solid. 

(a) Determine the differential constitutive law for the variant SLS. 

(b) Then, the creep compliance function $J_c(t)$ and hence, the relaxation function $G_r(t)$.

(c) How do the coefficients in the two variants of the standard solid model relate to each other?

\bigskip
\bigskip
\bigskip
\subsection*{3--3. \textbf{Frequency response of a 5-term analog model} [4 pts].}
You have a five-parameter fit $G_r(t) = C_r (200 e^{-2t} + 100 e^{-t} + 10)$ that describes the relaxation behavior of a real material. 

(a) Draw the equivalent mechanical analog model for this fit.

(b) Determine the functional forms for the storage and loss moduli, and create a semi-log plot of the loss tangent $(\tan\delta)$ over a domain of relevant frequency orders $(\log \omega)$. 

\newpage
\subsection*{3--4. \textbf{Fractional response} [4 pts].}

This question will be best approached numerically, using e.g. Matlab or Mathematica. 

Fractional order models can be used to show relaxation that does not follow the classic ``S-curve'' Debye relaxation function for $G_r(t)$ vs. $\log t$. 

Starting from a Kelvin-Voigt-type fractional model with the functional form of 
\begin{equation*}
    G_r(t) = \left[10 + 2\left(\frac{t}{0.2} \right)^{-\alpha}\right] \mathcal{H}(t),
\end{equation*}
plot the stress and strain responses of this solid over time (i.e., plot $\sigma(t,\alpha)$ and $\varepsilon(t,\alpha)$ on separate plots for each part) for a range of values of $0<\alpha<1$ to (a) a step strain, (b) a step stress of only length $t=5$, and (c) another stress function entirely of your choice. 

As a suggestion, you could consider values spaced symmetrically around zero on the logistic distribution, which is defined as $\textrm{logit}(\alpha) = \log\left(\frac{\alpha}{1-\alpha} \right)$. 
Picking e.g., logit($\alpha$)$=0$ corresponds to $\alpha =0.5$, $\textrm{logit}(\alpha) =  1$ is $\alpha\approx0.73$, etc. 
I suggest sampling integers on a range of logit($\alpha$) $= -4 \textrm{~to~} 4$ to cover the full range from elastic to viscous response for the springpot.

\bigskip
\bigskip
\subsection*{3--5. \textbf{Rheology without a rheometer} [8 pts].}

You have a rubbery material of density $\rho$ for which you plan to characterize frequency-dependent viscoelastic behavior. 
The material you have can be made into a sphere of a wide range of sizes, from a radius of $R=1$ mm to $R=1$ m. 
You plan to drop each ball onto a rigid half-space from a height $h_0$, and can measure the rebound height $h(R)$ for each ball radius $R$. 

The impact duration for an elastic material is given by a Hertzian contact relation of
\begin{equation*}
    t_c = 5.21\frac{R}{c}\left(\frac{c}{\sqrt{2 g h_0}}\right)^{1/5} \approx 0.025R  \textrm{~~[s]}
\end{equation*}
where $c = 1000$ m/s represents the pressure wave speed in the material and the initial height $h_0$ is taken to be a consistent 0.01 m.

(a) How much energy per volume is dissipated by the material for each size of ball? 

(b) Using the Lissajous plot of $\sigma/|E^*|$ vs. $\varepsilon$, show that the approximate peak elastic energy stored in the ball during a half-cycle is $\frac{1}{2} B^2 \cos \delta$, where $B = \varepsilon_{\textrm{max}}$. 

(c) Determine an approximate expression for the energy dissipated by the ball during a drop event in terms of $A = \varepsilon_{\textrm{max}} \sin \delta$ and $B$. 

(d) Hence, determine $\tan\delta$ as a function of the rebound height, $h(R)$. 

(e) For what frequencies could you say this material is calibrated?
\setcounter{section}{1} % This causes the next section to be Appendix B

\section{Kinetics, Constitutive Laws, and Viscoelasticity I}
\label{PS2}

This set of example problems is due on October 3, 2025. 
As before, I request that you type up your responses in \LaTeX~ rather than write them out by hand. 

\medskip
\subsection*{2--1. \textbf{Balance of mass} [4 pts].} 
A large piece of polydimethylsiloxane (PDMS) of uniform density $\rho(\bm{x},t)$ has a central spherical bubble of time-evolving radius $R(t)$, initial radius $R_0$, and wall velocity of $\dot{R}$. 
The hole is subject to a uniform surface traction in the $\bm{e}^{(r)} \equiv \bm{e}_{\bm{r}}$ direction from an axisymmetric pressure, and maintains spherical symmetry over time. 

\medskip
The position of a point in the material can be written as $\bm{x} = r(R,t) \bm{e}_{\bm{r}}$ with reference position $\bm{X} = r_0(R_0)$, while the velocity of that point can be written as $\bm{v}(r,t) = v_r \bm{e}_{\bm{r}}$.

\medskip
Using the conservation of mass equation, show that the material must satisfy
%\begin{equation}
%\rho_{,t} + (\rho v_i)_{,i} = 0,
%\end{equation}
\begin{equation*}
\rho_{,t}+ \rho_{,r} v_r + \frac{\rho}{r} (v_r + r v_{r,r}) = 0,
\end{equation*}
and hence, show that an assumption of incompressibility for PDMS results in 
\begin{equation*}
v_r(r,t) = \frac{R^2 \dot{R}}{r^2}.
\end{equation*}

\subsection*{\textbf{2-1 Solution:}}
(\textit{i}) Conservation of mass states:
\begin{gather*}
\frac{\partial\rho}{\partial t}+\nabla\cdot(\rho \bm{v})=0    
\end{gather*}
As we are working in spherical, and the velocity is stated to only have radial dependence, it is useful to define the divergence in spherical which is
\begin{gather*}
    \nabla\cdot(\bullet)=\frac{1}{r^2}\frac{\partial}{\partial r}(r^2(\bullet))+\frac{1}{r\text{sin}\theta}\frac{\partial}{\partial\theta}((\bullet)\text{sin}\theta)+\frac{1}{r\text{sin}\theta}\frac{\partial(\bullet)}{\partial\phi}
\end{gather*}
Applying this definition to the conservation of mass in conjunction with the sole radial dependence of velocity, we get
\begin{gather*}
    \rho_{,t}+\nabla\cdot(\rho\bm{v})=\rho_{,t}+\frac{1}{r^2}\frac{\partial}{\partial r}(r^2\rho v_r)=
    \rho_{,t}+\frac{2}{r}\rho v_r+ \rho_{,r}v_r+\rho v_{r,r}=0
\end{gather*}
Rearranging, we get
\begin{gather*}
    \rho_{,t}+\rho_{,r}v_r+\frac{2}{r}\rho v_r+\rho v_{r,r}=0\\
    \rho_{,t}+\rho_{,r}v_r+\frac{\rho}{r}(2 v_r+r v_{r,r})=0
\end{gather*}
We have thus demonstrated that a material must satisfy the provided equation via the conservation of mass equation.

(\textit{ii}) We begin with the conservation of mass once more. However, assuming incompressibility, $\rho$ is constant. Thus,
\begin{gather*}
    \frac{\partial\rho}{\partial t}+\nabla\cdot(\rho \bm{v})=\nabla\cdot \bm{v}=0 \\
    \frac{1}{r^2}\frac{\partial}{\partial r}(r^2v_r)=0\Rightarrow r^2v_r=C=\text{constant}    
\end{gather*}
We know that at $r=R$, $v_r=\dot{R}$. Then,
\begin{gather*}
    r^2v_r=C=R^2\dot{R}\\
    v_r=v_r(r,t)=\frac{R^2\dot{R}}{r^2}
\end{gather*}
We have hence shown the desired expression for $v_r$.

\medskip
\subsection*{2--2. \textbf{Balance of momenta} [4 pts].} A spherical hydrogel body $\mathcal{B}$ with a linear density gradient is currently submerged in water as depicted in the figure. 
The sphere has coordinates $\bm{x}$ in a region $\Omega$ with position-dependent density $\rho(\bm{x})$. 

\begin{figure}[H]
\vspace{-2em}
\centering
\includegraphics[width=3in]{instr-figures/PS2-Q1.pdf}
\caption{\small{Hydrogel sphere with a linear density gradient submerged in water. The water has a density $\rho_w$, while the sphere has a density at its leftmost point of $\rho_w/2$ and at its rightmost point of $3\rho_w/2$.}}
\end{figure}

\vspace{-1em}
The surface traction $\bm{t}(\bm{x},\hat{\bm{n}})$ acting on $\mathcal{B}$ is given by 
\begin{equation*}
\bm{t}(\bm{x},\hat{\bm{n}}) = -\rho_w g x_3 \hat{\bm{n}},
\end{equation*}
where $\hat{\bm{n}}$ is the outer unit normal to the surface $\partial \Omega_t$ and $\rho_w$ is the (constant) density of water and $g$ is the acceleration due to gravity. 

\medskip
(a) Determine the net force and moment acting on $\mathcal{B}$ via volume integrals.

\medskip
(b) Under what \textit{two} conditions is $\mathcal{B}$ in static equilibrium?

\subsection*{\textbf{2-2 Solution:}}
(\textit{a}) We will take the center of the sphere as our origin. Beginning with the balance of linear momentum.
\begin{gather*}
    \int_{\partial\Omega_t}\bm{t}d\bm{A_x}+\int_\Omega\bm{b}d\bm{V_x}=\int_\Omega\rho\bm{a}d\bm{V_x}=\text{net force}
\end{gather*}
We may convert the surface integral of the traction into a volume integral via the divergence theorem.
\begin{gather*}
    \bm{t}=-\rho_wgx_3\hat{\bm{n}}=\bm{\sigma}\cdot\hat{\bm{n}}\quad\therefore\quad\bm{\sigma}=-\rho_w gx_3 \bm{I}
\end{gather*}
Now, via the divergence theorem, we have
\begin{gather*}
    \int_\Omega\nabla_{\bm{x}}\cdot\bm{\sigma}d\bm{V_x}+\int_\Omega\bm{b}d\bm{V_x}=\int_\Omega\rho\bm{a}d\bm{V_x}
\end{gather*}
Our only body force is the weight of the hydrogel, which can be written as
\begin{gather*}
    \bm{b}=\rho(\bm{x})g\bm{e}_3=\rho_w(1+\frac{x_2}{2R})g\bm{e}_3
\end{gather*}
We can solve each of these integrals separately.
\begin{gather*}
    \int_\Omega\nabla_{\bm{x}}\cdot\bm{\sigma}d\bm{V_x}= \int_{\Omega}-\rho_w g \bm{e}_3d\bm{V_x}=-\frac{4}{3}\pi R^3\rho_wg\bm{e}_3\\
    \int_\Omega\bm{b}d\bm{V_x}=\int_{\Omega}\rho(\bm{x})g\bm{e}_3d\bm{V_x}=\rho_{avg}g\bm{e}_3\int_{\Omega} d\bm{V_x}
\end{gather*}
Since the density profile of this sphere is linear, one can easily find the average density to be $\rho_{avg}=\rho_w$. Thus,
\begin{gather*}
    \int_\Omega\bm{b}d\bm{V_x}=\rho_w g \bm{e}_3\int_{\Omega}d\bm{V_x}=\frac{4}{3}\pi R^3\rho_wg\bm{e}_3.
\end{gather*}
Adding these integrals, we see that
\begin{gather*}
    \text{net force }=-\frac{4}{3}\pi R^3\rho_wg\bm{e}_3+\frac{4}{3}\pi R^3\rho_wg\bm{e}_3=\bm{0}
\end{gather*}

Similarly, we find the net moment starting with the angular momentum balance.
\begin{gather*}
    \int_{\partial\Omega_t}\bm{x}\times\bm{t}d\bm{A_x}+\int_\Omega\bm{x}\times\bm{b}d\bm{V_x}=\int_\Omega\bm{x}\times(\rho\bm{a}d)\bm{V_x}=\text{net moment}
\end{gather*}
First, we find the moment due to the traction as
\begin{gather*}
    \int_{\partial\Omega_t}\bm{x}\times\bm{t}dA\bm{_x}=\int_{\partial\Omega_t}\bm{x}\times(\hat{\bm{n}}\cdot\bm{\sigma})dA\bm{_x}=\int_{\partial\Omega_t}\epsilon_{ijk}x_j n_p \sigma_{pk}\bm{e}_idA\bm{_x}\\
    =\int_{\Omega}\epsilon_{ijk}\frac{\partial}{\partial x_p}(x_j \sigma_{pk})\bm{e}_idV\bm{_x}\\
    =\int_{\Omega}\epsilon_{ijk}\frac{\partial}{\partial x_p}(-\rho_w g x_j x_3\delta_{pk})\bm{e}_idV\bm{_x}\\
    =\int_{\Omega}\epsilon_{ijk}\frac{\partial}{\partial x_k}(-\rho_w g x_j x_3)\bm{e}_idV\bm{_x}\\
    =-\rho_w g\int_{\Omega}\epsilon_{ijk}(\delta_{jk} x_3+x_j\delta_{3k})\bm{e}_idV\bm{_x}\\
    =-\rho_w g\int_{\Omega}\epsilon_{ijk}(x_j\delta_{3k})\bm{e}_idV\bm{_x}\\
    =-\rho_w g\int_{\Omega}\epsilon_{ij3}(x_j)\bm{e}_idV\bm{_x}\\
    =-\rho_w g\int_{\Omega}(x_2\bm{e}_1-x_1\bm{e}_2)dV\bm{_x}
\end{gather*}
It is now advantageous to convert this integral into spherical coordinates, leading to
\begin{gather*}
    \int_{\partial\Omega_t}\bm{x}\times\bm{t}dA\bm{_x}=-\rho_w g\int_{\Omega}(x_2\bm{e}_1-x_1\bm{e}_2)dV\bm{_x}=-\rho_w g\int_{\Omega}(r\text{ cos}(\phi)\bm{e}_1-r\text{sin}(\phi)\text{sin}(\theta)\bm{e}_2)dV\bm{_x}\\
    =-\rho_w g\int_0^R\int_0^{\pi}\int_0^{2\pi}(r^3\text{sin}(\phi)\text{ cos}(\phi)\bm{e}_1-r^3\text{sin}^2(\phi)\text{sin}(\theta)\bm{e}_2)d\theta d\phi dr\\
    =-2\pi\rho_w g\int_0^R\int_0^{\pi}r^2\text{sin}(\phi)\text{ cos}(\phi)\bm{e}_1 d\phi dr=0
\end{gather*}
Thus, the traction does not contribute to the net moment on the sphere at all. Now, we follow similar steps to find the body force's contribution to the net moment acting on the sphere.
\begin{gather*}
    \int_\Omega\bm{x}\times\bm{b}dV\bm{_x}=\int_{\Omega}\epsilon_{ijk}x_jb_k\bm{e}_idV\bm{_x}=\int_{\Omega}\epsilon_{ij3}x_j\rho(\bm{x})g\bm{e}_idV\bm{_x}\\
    =g\int_{\Omega}\rho_w(1+\frac{x_2}{2R})(x_2\bm{e}_1-x_1\bm{e}_2)dV\bm{_x}
\end{gather*}
At this point, it is once more easier to convert this to a spherical volume integral.
\begin{gather*}
    \int_\Omega\bm{x}\times\bm{b}dV\bm{_x}=\rho_wg\int_0^{2\pi}\int_0^{\pi}\int_0^R(1+\frac{r\text{ cos}(\phi)}{2R})(r\text{ cos}(\phi)\bm{e}_1-r\text{ sin}(\phi)\text{sin}(\theta)\bm{e}_2)r^2\text{sin}(\phi)dr d\phi d\theta\\
    =\rho_wg\int_0^{2\pi}\int_0^{\pi}\int_0^R[(r^3\text{cos}(\phi)\text{ sin}(\phi)+\frac{r^4}{2R}\text{cos}^2(\phi)\text{sin}(\phi))\bm{e}_1\\-(r^3\text{sin}^2(\phi)\text{sin}(\theta)+\frac{r^4}{2R}\text{sin}^2(\phi)\text{cos}(\phi)\text{sin}(\theta))\bm{e}_2]dr d\phi d\theta\\
    =\rho_w g\int_0^{2\pi}\int_0^{\pi}[(\frac{R^4}{4}\text{cos}(\phi)\text{ sin}(\phi)+\frac{R^4}{20}\text{cos}^2(\phi)\text{sin}(\phi))\bm{e}_1\\
    -(\frac{R^4}{4}\text{sin}^2(\phi)\text{sin}(\theta)+\frac{R^4}{20}\text{sin}^2(\phi)\text{cos}(\phi)\text{sin}(\theta))\bm{e}_2] d\phi d\theta\\
\end{gather*}
Note here that we may remove terms in this integral that we know will become 0 when we integrate from 0 to $2\pi$ and from 0 to $\pi$ w.r.t. $\theta$ and $\phi$ respectively.
\begin{gather*}
    =\rho_w g\int_0^{2\pi}\int_0^{\pi}[\frac{R^4}{20}\text{cos}^2(\phi)\text{sin}(\phi)\bm{e}_1] d\phi d\theta\\
    =\frac{2\pi\rho_wgR^4}{20}\bm{e}_1\int_o^{\pi}\text{cos}^2(\phi)\text{sin}(\phi) d\phi\\
    =\frac{\pi\rho_wgR^4}{15}\bm{e}_1.
\end{gather*}
Thus, our total moment is
\begin{gather*}
    \text{net moment }=\frac{\pi\rho_wgR^4}{15}\bm{e}_1.
\end{gather*}

(\textit{b}) The body will be in static equilibrium under the conditions of the net force acting on the body being zero and the net moment acting on the body being zero. In its current state, $\mathcal{B}$ only satisfies the former, as we found the net moment to be nonzero. This makes sense intuitively, as due to the asymmetric mass distribution in the sphere, the center of mass sits to the right (toward the positive $\bm{e}_2$ axis) of the body's geometric center. Thus, the force of gravity would be expected to induce a moment on the sphere, causing it to rotate about the $\bm{e}_1$ axis as we predicted. If the sphere was rotated $90^{\circ}$ about this axis, then it would be in static equilibrium.

% \subsection*{2--1. \textbf{Balance of momentum} [4 pts].} A Cauchy stress field in a material has a matrix of scalar components in the 3D basis $\{\bn{e}_i\}$:
% \begin{equation}
% \bm{\sigma} = \begin{bmatrix}
% 4x_1 x_3 & 0 & -2 x_3^2 \\
% 0 & 1 & 2 \\
% -2 x_3^2 & 2 & 3 x_1^2
% \end{bmatrix} \textrm{~(MPa)}
% \end{equation}
% where the material originally on a domain $\bm{X} \in \Omega_0$ is mapped to a new domain $\bm{x}\in \Omega$ (with units of meters). 

% \medskip
% (a) For the static case with no applied body force, is this stress field in equilibrium?

% \medskip
% (b) At a position vector $\bm{x}_1 = 2\bm{e}_1 +  \bm{e}_2 + \bm{e}_3 $, determine the traction, $\bm{t}$, caused by the stress tensor at a cut plane given by the equation $x_1 + x_2 - x_3 = 2$. 

% \medskip
% (c) Find the magnitudes of the shear and normal traction on this plane at the point $\bm{x}_1$. 

% \medskip
% (d) Computationally determine the principal stresses and associated directions at the given point. Why, in general, can a principal stress be negative but a principal stretch cannot?


\bigskip
\subsection*{2--3. \textbf{Viscoelastic data} [4 pts].} 
Stress relaxation isochrones for a compliant viscoelastic material are shown in the figure below.  

\begin{figure}[H]
\vspace{-1em}
\centering
\includegraphics[scale = 1.5]{instr-figures/PS2-Q3.pdf}
\caption{\small{Stress (Pa) vs. strain ($-$) for a soft viscoelastic material.}}
\end{figure}

\vspace{-1em}
(a) Are these isochrones from a material which we can describe with linear viscoelasticity? If not, why not, and if so, under what approximate regimes would this assumption be valid? 

\medskip
(b) Estimate the creep relaxation function $J_c$ for stress values of 100 and 250 kPa. Isochrones are shown at times of 2, 5, 10, 20, and 40 seconds.   
% This is a placeholder for the example problems from the second problem set. 
% You'll replace this file with the one I supply on canvas. 

\subsection*{\textbf{2-3 Solution:}}
(\textit{a}) If the isochrones appear to be approximately linear in a strain regime, then it is safe to describe the material behavior with linear viscoelasticity (LVE). In general, as these isochrones are clearly nonlinear, we cannot describe this material with linear viscoelasticity. This nonlinearity implies that the creep compliance $J_c$ for this material is dependent on not just time but on stress $\sigma$ as well (i.e. $J_c=J_c(t,\sigma)$. To describe the material with LVE we require $J_c=J_c(t)$, exemplified by linear isochrones.

So, this assumption would be valid approximately for the strain regime from $\sim 0-1.8\cdot10^{-3}$ and again for the regime from $\sim 3.3-4.0\cdot10^{-3}$.

(\textit{b}) We can approximate values of $J_c$ at the given isochrone times via $J_c(t)=\frac{\varepsilon(t)}{\sigma_0}$. These values are displayed in Table \ref{tab:2-3Sol}. 

\begin{table}[h!]
    \centering
    \begin{tabular}{|c|c|c|c|c|}
    \hline
        t & $\sigma_0=100$ Pa &  & $\sigma_0=250$ Pa & \\
        \hline
         & $\varepsilon$ & $J_c$ & $\varepsilon$ & $J_c$\\
         \hline
        2 & .44E-3 & .44E-5 & 1.2E-3 & 4.8E-6\\
        5 & .53E-3 & .53E-5 & 1.45E-3 & 5.8E-6\\
        10 & .67E-3 & .67E-5 & 1.8E-3 & 7.2E-6\\
        20 & .81E-3 & .81E-5 & 2.4E-3 & 9.6E-6\\
        40 & 1.0E-3 & 1.0E-5 & 3.0E-3 & 1.2E-5\\
        \hline
    \end{tabular}
    \caption{Approximated values for estimating $J_c$}
    \label{tab:2-3Sol}
\end{table}

Using Matlab, we may plot ln($J_c$) vs. ln($t$), linear fit this ln-ln plot, and then we may use these constants to solve for $J_c$. Doing this, we obtain the following fits.
\begin{gather*}
    \text{For }\sigma_0=100\text{ Pa, } J_c=3.52\cdot10^{-6}t^{.2788}.\\
    \text{For }\sigma_0=250\text{ Pa, } J_c=3.67\cdot10^{-6}t^{.3145}.
\end{gather*}

\bigskip
\subsection*{2--4. \textbf{Impulsive stresses} [4 pts].}

(a) Say that instead of a step load, we apply $\sigma(t) = A \delta(t)$ to an unknown linear viscoelastic material. 
Determine the strain history $\epsilon(t)$, first as a general function of the creep relaxation function $J_c(t)$, and then for a Kelvin-Voigt solid. 

(b) Now, consider a rapid load followed by a rapid reverse load by applying a doublet function of stress, i.e. $\sigma(t) = B \psi(t)$. 
What is the strain function $\epsilon(t)$ in terms of $J_c(t)$ and for a Kelvin-Voigt material now? 

\subsection*{\textbf{2-4 Solution:}}
(\textit{a}) First, we can find the strain history $\varepsilon(t)$ as a general function of the creep relaxation as follows.
\begin{align*}
    \varepsilon(t)&=\int_{0^-}^tJ_c(t-\tau)\frac{d\sigma(\tau)}{d\tau}d\tau\\
    &=A\int_{0^-}^tJ_c(t-\tau)\psi(\tau)d\tau
\end{align*}

We then can find this using the constitutive law for a Kelvin-Voigt material as follows.
\begin{gather*}
    \eta\dot{\varepsilon}+E\varepsilon=\sigma=A\delta\\
    \dot{\varepsilon}\text{ exp}( \frac{Et}{\eta})+\frac{E}{\eta}\varepsilon\text{ exp}(\frac{Et}{\eta})=\frac{A}{\eta}\delta\text{ exp}(\frac{Et}{\eta})\\
    \frac{d}{dt}[\text{exp}(\frac{Et}{\eta})\varepsilon]=\frac{A}{\eta}\delta\text{ exp}(\frac{Et}{\eta})\\
    \text{exp}(\frac{Et}{\eta})\varepsilon=\frac{A}{\eta}[H(t)\text{ exp}(\frac{Et}{\eta})]+C\\
    \varepsilon(t)=\frac{A}{\eta}H(t)+C\text{ exp}(\frac{-Et}{\eta})
\end{gather*}
where $C$ is a constant.

(\textit{b}) We solve for the strain in a similar manner as part (a).
\begin{align*}
    \varepsilon(t)&=\int_{0^-}^tJ_c(t-\tau)\frac{d\sigma(\tau)}{d\tau}d\tau\\
    &=B\int_{0^-}^tJ_c(t-\tau)\delta^{('')}(\tau)d\tau=B\int_{0^-}^tJ_c(\tau)\delta^{('')}(t-\tau)d\tau
\end{align*}
Due to the nature of the dirac delta family of "functions", which are equal to 0 everywhere but at 0, we may state
\begin{align*}
    \varepsilon(t)&=B\int_{0^-}^tJ_c(\tau)\delta^{('')}(t-\tau)d\tau\\
    &=B\int_{-\infty}^{\infty}J_c(\tau)\delta^{('')}(t-\tau)d\tau\\
    &=B\cdot(-1)^2\frac{d^2J_c}{dt^2}|_{t=\tau}\\
    \varepsilon(t)&=B\frac{d^2J_c}{dt^2}|_{t=\tau}
\end{align*}

We then can find this using the constitutive law for a Kelvin-Voigt material as follows.
\begin{gather*}
    \eta\dot{\varepsilon}+E\varepsilon=\sigma=B\psi(t)\\
    \dot{\varepsilon}\text{ exp}( \frac{Et}{\eta})+\frac{E}{\eta}\varepsilon\text{ exp}(\frac{Et}{\eta})=\frac{B}{\eta}\psi\text{ exp}(\frac{Et}{\eta})\\
    \frac{d}{dt}[\text{exp}(\frac{Et}{\eta})\varepsilon]=\frac{B}{\eta}\psi\text{ exp}(\frac{Et}{\eta})\\
    \text{exp}(\frac{Et}{\eta})\varepsilon=\frac{B}{\eta}[(-1)^2\frac{E^2}{\eta^2}\text{exp}(\frac{Et}{\eta})]_{t=0}+C\\
    \varepsilon(t)=(\frac{BE^2}{\eta^3}+C)\text{ exp}(\frac{Et}{\eta})
\end{gather*}
where C is a constant.

\bigskip
\subsection*{2--5. \textbf{Two-element models} [8 pts].}

Dynamic mechanical analysis (DMA) is a common technique for characterizing viscoelasticity. 
DMA conventionally involves application of a sinusoidal displacement to the top surface of a sample at a controllable temperature. 
Often, the user puts the sample into initial compression, and follows with the sinusoidal profile. 
A cylindrical sample of height $h$ and diameter $d$ is placed between two plates.
The DMA then quickly puts the sample into compression by moving its top plate downward by a displacement $d$, and then oscillates sinusoidally between positions $0$ and $2d$ at a frequency $\omega$.

\medskip
(a) Using the constitutive law for a Kelvin-Voigt material, determine the stress $\sigma(t)$ exerted by the platens to cause the applied strain. 

\medskip
(b) The resulting stress lags behind the strain by an phase $\delta$, as in $\sin(\omega t + \delta)$. 
Commonly this is reported as the ``tangent loss", or $\tan\delta$, for a material. 
What is the value of $\tan\delta$ for this particular Kelvin-Voigt model?

\medskip
(c) Say instead of prescribing the strain $\epsilon(t)$, we instead prescribed the stress, $\sigma(t) = - \sigma_0 - \sigma_0 \sin(\omega t)$. 
Determine the strain $\epsilon(t)$ for this prescribed stress.

\medskip
(d) Prove that the tangent loss function $\tan\delta$ is identical between the two loading methods.

\subsection*{\textbf{2-5 Solution:}}
(\textit{a}) The constitutive law for a Kelvin-Voigt model is stated as:
\begin{gather*}
    \sigma(t)=\eta\dot{\varepsilon}(t)+E\varepsilon(t)
\end{gather*}
where $\eta$ and $E$ are material constants. Taking compression as negative strain, we can write the prescribed strain condition as
\begin{gather*}
    \varepsilon(t)=-\frac{d}{h}(1+\text{sin}(\omega t))
\end{gather*}
and we can thus find $\dot{\varepsilon}(t)$ as
\begin{gather*}
    \dot{\varepsilon}(t)=-\frac{\omega d}{h}\text{cos}(\omega t)
\end{gather*}
Plugging these into the Kelvin-Voigt material constitutive law, we get
\begin{gather*}
    \sigma(t)=-\frac{\eta\omega d}{h}\text{cos}(\omega t)-\frac{Ed}{h}-\frac{Ed}{h}\text{sin}(\omega t)\\
    \sigma(t)=-\frac{d}{h}E-\frac{d}{h}(E\text{sin}(\omega t)+\eta\omega\text{cos}(\omega t))
\end{gather*}

(\textit{b}) We may combine the sine and cosine terms into one sine term using the following phase shift method.
\begin{gather*}
    A\text{sin}(\omega t)+B\text{cos}(\omega t)=\sqrt{A^2+B^2}(\text{sin}(\omega t + \delta))\\
    \text{where}\quad\text{tan}(\delta)=\frac{A}{B}
\end{gather*}
So, rewriting our stress function, we find
\begin{gather*}
    \sigma(t)=-\frac{d}{h}E-\frac{d}{h}\sqrt{E^2+(\eta\omega)^2}(\text{sin}(\omega t+ \text{arctan}(\frac{E}{\eta\omega}))
\end{gather*}
Thus, the value of the tangent loss for this particular Kelvin-Voigt model is $\frac{E}{\eta\omega}$.

(\textit{c}) Using the same constitutive law stated before, we can find strain as follows.
\begin{gather*}
    \sigma(t)=-\sigma_0-\sigma_0\text{sin}(\omega t)=\eta\dot{\varepsilon}+E\varepsilon\\
    -\frac{\sigma_0}{\eta}(1+\text{sin}(\omega t))\text{exp}(\frac{Et}{\eta})=\dot{\varepsilon}(\text{exp}(\frac{Et}{\eta}))+E\varepsilon(\text{exp}(\frac{Et}{\eta}))\\
    -\frac{\sigma_0}{\eta}\text{exp}(\frac{Et}{\eta})-\frac{\sigma_0}{\eta}\text{sin}(\omega t)\text{exp}(\frac{Et}{\eta})=\frac{d}{dt}[\varepsilon(t)(\text{exp}(\frac{Et}{\eta}))]\\
    -\frac{\sigma_0}{E}(\text{exp}(\frac{Et}{\eta}))-\frac{\sigma_0}{\eta}[\frac{\text{exp}(\frac{Et}{\eta})}{(\frac{E}{\eta})^2+\omega^2}(\frac{E}{\eta}\text{sin}(\omega t)-\omega\text{cos}(\omega t))]+C=\varepsilon(t)(\text{exp}(\frac{Et}{\eta}))\\
    \varepsilon(t)=-\frac{\sigma_0}{E}(\text{exp}(\frac{Et}{\eta}))-\frac{\sigma_0}{\eta}[\frac{\frac{E}{\eta}\text{sin}(\omega t)-\omega\text{cos}(\omega t)}{(\frac{E}{\eta})^2+\omega^2}]+C(\text{exp}(\frac{-Et}{\eta}))
\end{gather*}
C is a constant that may be found from initial conditions. We may further simplify this as was done using the phase shift equations from part (\textit{b}). Doing so, we get
\begin{gather*}
    \varepsilon(t)=-\frac{\sigma_0}{E}-\frac{\sigma_0}{\eta}[\frac{\text{sin}(\omega t+\text{arctan}(\frac{-E}{\eta\omega}))}{\sqrt{(\frac{E}{\eta})^2+\omega^2}}]+C(\text{exp}(\frac{-Et}{\eta})).
\end{gather*}
If we had an initial condition, such as strain being zero at $t=0$, then we could find a value for C. In this case, if $\epsilon(0)=0$, we'd get
\begin{gather*}
    C=\frac{\sigma_0}{E}+\frac{\sigma_0\text{ sin(arctan}(\frac{-E}{\eta\omega})}{\eta\sqrt{(\frac{E}{\eta})^2+\omega^2}}.
\end{gather*}
(\textit{d}) From the steps taken in (\textit{c}), we thus see the magnitudes of the tangent loss function are equal between these two loading methods as
\begin{gather*}
    |\text{tan}(\delta_1)|=|\text{tan}(\delta_2)|=\frac{E}{\eta\omega}.
\end{gather*}
%
\section*{Examples I. Mathematical Preliminaries (due \textcolor{red}{Sept 19})}
\textcolor{red}{(Rev note: v2)}
\label{PS1}

This set of example problems is due on September 17, 2025. 
I request that you type up your responses in \LaTeX~ rather than write them out by hand. 
The primary reason is to become better acquainted with writing up mechanics in archival format. 
If you have diagrams, plots, etc., please add them as attached figures using the \texttt{includegraphics} command. 

\bigskip
\subsection*{1--1. \textbf{Convolutional integrals} [4 pts].} The response of a 1D viscoelastic material to an applied forcing function is given using a convolutional integral:
\begin{equation}
    \varepsilon(t) = \int_0^t J(t-\tau) \frac{d\sigma(\tau)}{d\tau} d\tau,
\end{equation}
where $\varepsilon(t)$ is the time-dependent strain response, $\sigma(t)$ is the prescribed stress function, and $J(t)$ is the material compliance, assumed to not depend on the level of stress applied. 
Say we have a compliance function 
\begin{equation}
    J(t) = J_\infty + (J_0-J_\infty)\exp[-t/\tau_c],
\end{equation}
where $J_0, J_\infty, \tau_c$ are all constants $\in \mathbbm{R}$. 
We subject this material to two different loading profiles: (a) step load $\sigma_1(t) = \sigma_0 H(t)$ and (b) a sinusoidal load $\sigma_2(t) = \sigma_0  \sin(\omega t)$, where $\sigma_0$ and $\omega$ are also constant, and $H(t)$ is the step function. 

Determine the corresponding Laplace transforms of the strain functions $\mathcal{L}\{\varepsilon_1(t)\}$ and $\mathcal{L}\{\varepsilon_2(t)\}$. 

\textit{\textbf{Dare mode:}} If you have taken complex analysis, you can compute the inverse transform of polynomial forms because this type of model yields terms with simple poles (i.e. linear in $s$). 
The formulae for this (see e.g. \cite{rileyMathematicalMethodsPhysics2006} Chs. 24 and 25) are:
\begin{equation*}
    \textrm{Residue for simple poles: } R(f(s),s_0) = \lim\limits_{s\rightarrow s_0} \left[ (s-s_0) f(s) \right],
\end{equation*}
and you multiply each residue by the shift from zero, i.e.,
\begin{equation*}
    f(t) = \mathcal{L}^{-1}\{F(s)\} = \sum \left( \textrm{residues of } F(s)e^{s_0 t} \textrm{ at all poles } s_0 \right)
\end{equation*}
\textit{If you dare}, determine the corresponding strain histories $\varepsilon_1(t)$ and $\varepsilon_2(t)$. The first is relatively straightforward; the latter has complex poles and more terms. 

\subsection*{\textbf{1-1 Solution:}}
Recall that
\begin{equation*}(f*g)(t)=\int_0^tf(\tau)g(t-\tau)d\tau.\end{equation*}
If
\begin{equation*}
    f(\tau)=\frac{d\sigma(\tau)}{d\tau}\quad\text{and}\quad g(t-\tau)=J(t-\tau),
\end{equation*}
then
\begin{equation*}\varepsilon(t)=(\sigma'*J)(t)=\int_0^t \frac{d\sigma(\tau)}{d\tau}J(t-\tau) d\tau\end{equation*}

So, we may then state that
\begin{equation*}
    \mathcal{L}\{\varepsilon(t)\}=\mathcal{L}\{(\sigma'*J)(t)\}=\mathcal{L}\{{\sigma'(t)}\}\cdot\mathcal{L}\{J(t)\}
\end{equation*}
For $J(t)$, we see
\begin{equation*}
    \mathcal{L}\{J(t)\}=J(s)=\frac{J_{\infty}}{s}+\frac{(J_0-J_{\infty})}{s+\frac{1}{\tau_c}}
\end{equation*}
(a) For $\varepsilon_1(t)$, $\sigma_1(t)=\sigma_0H(t)$. So,
\begin{equation*}
    \mathcal{L}\{\frac{d}{dt}\sigma_1(t)\}=\mathcal{L}\{\frac{d}{dt}(\sigma_0H(t))\}=s\mathcal{L}\{\sigma_0H(t)\}-\sigma_0H(0)=\sigma_0
\end{equation*}
Thus, $\mathcal{L}\{\varepsilon_1(t)\}=\sigma_0[\frac{J_{\infty}}{s}+\frac{(J_0-J_{\infty})}{s+\frac{1}{\tau_c}}]$.

\medskip
(b) For $\varepsilon_2(t)$, $\sigma_2(t)=\sigma_0\text{sin}(\omega t)$. So,
\begin{equation*}
    \mathcal{L}\{\frac{d}{dt}\sigma_2(t)\}=\mathcal{L}\{\frac{d}{dt}(\sigma_0\text{sin}(\omega t))\}=\sigma_0(s\mathcal{L}\{\text{sin}(\omega t)\}-\text{sin}(0))=\frac{s\sigma_0\omega}{s^2+\omega^2}
\end{equation*}
Thus, $\mathcal{L}\{\varepsilon_2(t)\}=(\frac{s\sigma_0\omega}{s^2+\omega^2})(\frac{J_{\infty}}{s}+\frac{(J_0-J_{\infty})}{s+\frac{1}{\tau_c}})$
%\newpage
\bigskip
\subsection*{1--2. \textbf{Index notation} [4 pts].} Let $\bm{p}, \bm{q}, \bm{r}, \bm{a}, \bm{b}$ be vector fields on $\mathbbm{R}^3$ and $\bn{Q}$ be a change-of-basis tensor on $\mathbbm{R}^3$. Show the following identities to be true using index notation. 

\begin{itemize}
    \item $\bm{p} \times (\bm{q} \times \bm{r}) = (\bm{r} \cdot \bm{p}) \bm{q} - (\bm{q} \cdot \bm{p}) \bm{r}$
    \item $(\bm{p} \times \bm{q}) \cdot (\bm{a} \times \bm{b}) = (\bm{p} \cdot \bm{a}) (\bm{q} \cdot \bm{b}) - (\bm{q} \cdot \bm{a})(\bm{p} \cdot \bm{b})$
    \item $(\bm{a} \otimes \bm{b})(\bm{p} \otimes \bm{q}) = \bm{a}\otimes\bm{q}(\bm{b} \cdot \bm{p}) $
    \item $\bn{Q}^\intercal\bm{a} \cdot \bn{Q}^\intercal\bm{b} = \bm{a}\cdot\bm{b} $
\end{itemize}

\subsection*{\textbf{1-2 Solution:}}
(\textit{i}) \begin{gather*}
    LHS =q\times r \Rightarrow \epsilon_{ijk}q_jr_k=w_i\\
    p\times(q\times r)=p\times w \Rightarrow\epsilon_{abc}p_bw_c=\epsilon_{abc}p_b\epsilon_{cjk}q_j r_k \\
    \epsilon_{abc}\epsilon_{cjk}=\epsilon_{cab}\epsilon_{cjk}=\delta_{aj}\delta_{bk}-\delta_{ak}\delta_{bj}\\
    (\delta_{aj}\delta_{bk}-\delta_{ak}\delta_{bj})(p_b q_j r_k)\\
    p_k r_k q_a - p_j q_j r_a = r_k p_k q_a - q_j p_j r_a \Rightarrow (\bm{r}\cdot\bm{p})\bm{q}-(\bm{q}\cdot\bm{p})\bm{r} = RHS
\end{gather*}

\medskip
(\textit{ii})\begin{gather*}
    LHS=(\bm{p} \times \bm{q}) \cdot (\bm{a} \times \bm{b})\Rightarrow\epsilon_{ijk}p_j q_k \epsilon_{imn}a_m b_n\\
    \epsilon_{ijk}p_j q_k \epsilon_{imn}a_m b_n=\delta_{jm}\delta_{kn}-\delta_{jn}\delta_{km}\\
    \epsilon_{ijk} \epsilon_{imn}p_j q_k a_m b_n=(\delta_{jm}\delta_{kn}-\delta_{jn}\delta_{km})p_j q_k a_m b_n=p_ja_jq_kb_k-q_ma_mp_nb_n\\
    p_ja_jq_kb_k-q_ma_mp_nb_n \Rightarrow (\bm{p} \cdot \bm{a}) (\bm{q} \cdot \bm{b}) - (\bm{q} \cdot \bm{a})(\bm{p} \cdot \bm{b})=RHS
\end{gather*}

\medskip
(\textit{iii})\begin{gather*}
    LHS = (\bm{a} \otimes \bm{b})(\bm{p} \otimes \bm{q}) \Rightarrow a_i b_j (\bm{e}^{(i)}\otimes\bm{e}^{(j)})p_k q_l (\bm{e}^{(k)}\otimes\bm{e}^{(l)})=a_i b_j p_k q_l \delta_{jk}(\bm{e}^{(i)}\otimes\bm{e}^{(l)})=\\
    a_i b_k p_k q_l (\bm{e}^{(i)}\otimes\bm{e}^{(l)})=a_i q_l (\bm{e}^{(i)}\otimes\bm{e}^{(l)}) b_k p_k \Rightarrow (\bm{a}\otimes\bm{q})(\bm{b} \cdot \bm{p})=RHS
\end{gather*}

\medskip
(\textit{iv})\begin{gather*}
    LHS = \bm{Q}^\intercal\bm{a} \cdot \bm{Q}^\intercal\bm{b} \Rightarrow Q_{ij}^\intercal a_j Q_{ip}^\intercal b_p=\\
    Q_{ji} a_j Q_{pi} b_p = Q_{pi}Q_{ij}^\intercal a_j b_p = \delta_{pj}a_j b_p= a_p b_p \Rightarrow \bm{a}\cdot\bm{b} = RHS
\end{gather*}

\subsection*{1--3. \textbf{Tensors and vectors} [4 pts].}
The second-order projection tensors $\bn{P}_{\bm{n}}^{||}$ and $\bn{P}_{\bm{n}}^{\perp}$ are useful operators that take a vector $\bm{u}$ and map that vector to its part parallel and perpendicular to a vector $\bm{n}$, respectively. 

They are defined via:
\begin{equation*}
    \bm{u}_{||} = (\bm{u} \cdot \bm{n}) \bm{n} = (\bm{n} \otimes \bm{n}) \bm{u} = \bn{P}_{\bm{n}}^{||} \bm{u},
\end{equation*}
\begin{equation*}
    \bm{u}_{\perp} = \bm{u} - \bm{u}_{||} = (\bn{I} - \bm{n} \otimes \bm{n}) \bm{u} = \bn{P}_{\bm{n}}^{\perp} \bm{u}.
\end{equation*}

The projection tensors have properties
\begin{align*}
    \bn{P}_{\bm{n}}^{||} + \bn{P}_{\bm{n}}^{\perp} &= \bn{I} \\
    \left(\bn{P}_{\bm{n}}^{||} \right)^m &= \bn{P}_{\bm{n}}^{||} ~\forall ~m \in \mathbbm{Z}^+\\
    \left(\bn{P}_{\bm{n}}^{\perp} \right)^m &= \bn{P}_{\bm{n}}^{\perp} ~\forall ~m \in \mathbbm{Z}^+\\
    \bn{P}_{\bm{n}}^{||} \bn{P}_{\bm{n}}^{\perp} = \bn{P}_{\bm{n}}^{\perp} \bn{P}_{\bm{n}}^{||}  &= \bn{0}
\end{align*}

Using the projection tensors, show that $\bm{u} = (\bm{u} \cdot \bm{n}) \bm{n} + \bm{n} \times (\bm{u} \times \bm{n} )$.

\subsection*{\textbf{1-3 Solution:}}
(Note: we must assume $\bm{n}$ is a unit vector for projection tensors.) The first term is already equal to the component of $\bm{u}$ parallel to $\bm{n}$, (i.e. $(\bm{u}\cdot\bm{n})\bm{n}=\bm{u}_{||}$). So, we must now show $\bm{n}\times(\bm{u}\times\bm{n})= \bm{u}_{\perp}$.
\begin{gather*}
    \bm{u}\times\bm{n} \Rightarrow \epsilon_{iab}u_a n_b\\
    \bm{n}\times(\bm{u}\times\bm{n}) \Rightarrow \epsilon_{kji}n_j\epsilon_{iab}u_a n_b\\
    \epsilon_{kji}\epsilon_{iab}=\epsilon_{ikj}\epsilon_{iab}= \delta_{ka}\delta_{jb}-\delta_{kb}\delta_{ja}\\
    \epsilon_{kji}n_j\epsilon_{iab}u_a n_b=(\delta_{ka}\delta_{jb}-\delta_{kb}\delta_{ja})n_j n_b u_a=\\
    n_j n_j u_k - n_k n_j u_j \Rightarrow (\bm{n}\cdot\bm{n})\bm{u}-(\bm{n}\otimes\bm{n})\bm{u}=\bm{u}-(\bm{n}\otimes\bm{n})\bm{u}=(\bn{I}-\bm{n}\otimes\bm{n})\bm{u}=\bm{u}_{\perp}
\end{gather*}
Thus,
\begin{gather*}
    (\bm{u}\cdot\bm{n})\bm{n}+\bm{n}\times(\bm{u}\times\bm{n})= \bm{u}_{||}+\bm{u}_{\perp}=\bm{u}
\end{gather*}

\bigskip
\subsection*{1--4. \textbf{Vector and tensor calculus} [4 pts].} Show the following vector and tensor identities to be true using index notation:

\begin{itemize}
    \item $\gradX \times (\phi \bm{a}) = \phi \gradX \times \bm{a} + (\gradX\phi) \times \bm{a}$
    \item $\gradX (\bm{a} \cdot \bm{b}) = (\bm{a} \cdot \gradX) \bm{b} + (\bm{b} \cdot \gradX) \bm{a} + \bm{a} \times (\gradX \times \bm{b}) + \bm{b} \times (\gradX \times \bm{a})$
    \item $ (\bn{A} \bn{B}) \bn{:} \bn{C} = (\bn{A}^\intercal \bn{C})\bn{:} \bn{B} = (\bn{C} \bn{B}^\intercal)\bn{:} \bn{A}$
    \item Let $J = \det \bn{F}$. Show\footnote{It will help to use the expression for the determinant of a tensor in index notation!} that $\frac{\partial J}{\partial \bn{F}} = J \bn{F}^{-\intercal}$. 
    \end{itemize}

\subsection*{\textbf{1-4 Solution:}}
(\textit{i}) \begin{gather*}
    LHS = \gradX \times (\phi \bm{a}) \Rightarrow \epsilon_{ijk}\frac{\partial(\phi a_k)}{\partial X_j}=\epsilon_{ijk}(\frac{\partial\phi}{\partial X_j}a_k+\frac{\partial a_k}{\partial X_j}\phi)=\\
    \phi \epsilon_{ijk}\frac{\partial a_k}{\partial X_j}+\epsilon_{ijk}\frac{\partial\phi}{\partial X_j}a_k\Rightarrow \phi \gradX \times \bm{a} + (\gradX\phi) \times \bm{a} = RHS
\end{gather*}

(\textit{ii}) We will begin by expanding each term on the right in index notation.
\begin{gather*}
    (\bm{a} \cdot \gradX) \bm{b}\Rightarrow a_m\frac{\partial}{\partial X_m}b_i=a_m\frac{\partial b_i}{\partial X_m};\\
    (\bm{b} \cdot \gradX) \bm{a}\Rightarrow b_m\frac{\partial}{\partial X_m}a_i=b_m\frac{\partial a_i}{\partial X_m};\\
    \bm{a} \times (\gradX \times \bm{b})\Rightarrow \epsilon_{iqr}a_q\epsilon_{rjk}\frac{\partial b_k}{\partial X_j}=\epsilon_{riq}\epsilon_{rjk}a_q\frac{\partial b_k}{\partial X_j}=(\delta_{ij}\delta_{qk}-\delta_{ik}\delta_{jq})a_q\frac{\partial b_k}{\partial X_j}=a_k \frac{\partial b_k}{\partial X_i}- a_j \frac{\partial b_i}{\partial X_j};\\
    \bm{b} \times (\gradX \times \bm{a})\Rightarrow b_k \frac{\partial a_k}{\partial X_i}- b_j \frac{\partial a_i}{\partial X_j};
\end{gather*}
Swapping dummy indices, we see we may cancel the first and second terms of the original equation with corresponding parts of the expanded third and fourth terms. This leaves us with
\begin{gather*}
    a_k\frac{\partial b_k}{\partial X_i}+b_k\frac{\partial a_k}{\partial X_i}= \frac{\partial (a_k b_k)}{\partial X_i}\Rightarrow \gradX (\bm{a}\cdot\bm{b})
\end{gather*}
We have thus shown that the left hand side and right hand side of this statement are equal.

(\textit{iii})
For the middle part of this equation,
\begin{gather*}
    (\bn{A}^\intercal \bn{C})\bn{:} \bn{B}\Rightarrow A^{\intercal}_{ij}C_{jk}B_{ik}=A_{ji}B_{ik}C_{jk}\Rightarrow (\bn{A} \bn{B}) \bn{:} \bn{C} = LHS.
\end{gather*}
For the right side,
\begin{gather*}
    (\bn{C} \bn{B}^\intercal)\bn{:} \bn{A}\Rightarrow C_{ij}B^\intercal_{jk}A_{ik}=A_{ik}B_{kj}C_{ij}\Rightarrow (\bn{A} \bn{B}) \bn{:} \bn{C} = LHS.
\end{gather*}

(\textit{iv})
In index notation,
\begin{gather*}
    J = \frac{1}{6}\epsilon_{ijk}\epsilon_{pqr}F_{ip}F_{jq}F_{kr}
    \frac{\partial J}{\partial F_{ab}}=\frac{1}{6}\epsilon_{ijk}\epsilon_{pqr}\frac{\partial(F_{ip}F_{jq}F_{kr})}{\partial F_{ab}}
\end{gather*}
Using the product rule, this becomes
\begin{gather*}
    \frac{\partial J}{\partial F_{ab}}=\frac{1}{6}\epsilon_{ijk}\epsilon_{pqr}(\delta_{ai}\delta_{bp}F_{jq}F_{kr} + \delta_{aj}\delta_{bq}F_{ip}F_{kr} + \delta_{ak}\delta_{br}F_{ip}F_{jq})=\\
    \frac{1}{6}(\epsilon_{ajk}\epsilon_{bqr}F_{jq}F_{kr}+\epsilon_{iak}\epsilon_{pbr}F_{kr}F_{ip}+\epsilon_{ija}\epsilon_{pqb}F_{ip}F_{jq})
\end{gather*}
We can rewrite the order of the indices on the alternators such that we get
\begin{gather*}
    \frac{1}{6}(\epsilon_{ajk}\epsilon_{bqr}F_{jq}F_{kr}+\epsilon_{aki}\epsilon_{brb}F_{kr}F_{ip}+\epsilon_{aij}\epsilon_{bpq}F_{ip}F_{jq})
\end{gather*}
Swapping dummy indices, we see that these three terms are all equivalent. So, we end up with
\begin{gather*}
    \frac{\partial J}{\partial F_{ab}}=\frac{1}{2}\epsilon_{ajk}\epsilon_{bqr}F_{jq}F_{kr}.
\end{gather*}
In Bauer's continuum mechanics notes, he states that for a second order tensor $\bn{S}$, $S^{-\intercal}_{ab}=\frac{1}{2\text{det}\bn{S}}\epsilon_{ajk}\epsilon_{bqr}F_{jq}F_{kr}$. Thus, we see that our result can be written as
\begin{gather*}
    \frac{\partial J}{\partial F_{ab}}=\frac{1}{2}\epsilon_{ajk}\epsilon_{bqr}F_{jq}F_{kr}= F^{-\intercal}_{ab}\text{det}\bn{F}\\
    \Rightarrow \frac{\partial J}{\partial \bn{F}}=J\bn{F}^{-\intercal}
\end{gather*}

%\newpage
\bigskip
\subsection*{1--5. \textbf{Kinematics} [8 pts].} The Happy Gelatinous Cube (HGC, pictued) $\mathcal{G}$ exists on a domain of $\{-1\leq X_1 , X_3\leq1, 0\leq X_2 \leq 2\}$ at initial time $t=0$. 
At all times, the bottom surface of the HGC does not move. 
Its top surface moves sinusoidally in time at frequency $\omega$ by a maximum magnitude of $\alpha$. 
At maximum compression, points in the centers of the surfaces defined by outward normals $\bm{e}_1$ and $\bm{e}_3$ experience maximum displacements of magnitude $\beta$. 

\medskip
(a) Determine the deformation gradient tensor $[\bn{F}(\bm{X})]^{\bm{e}}$ for all $\bm{X}\in \mathcal{G}$. 
Describe any assumptions you make about the shape of the HGC as it deforms. 

\medskip
(b) Determine the stretch magnitude of a small fiber positioned at a height $X_2 = 1$ and oriented at an angle $\theta$ from the $\bm{e}_1$ axis \textcolor{red}{(in either the $\bm{e}_1- \bm{e}_2$ or $\bm{e}_1- \bm{e}_3$ plane)}. 

\medskip
(c) Determine the Lagrange-Green strain tensor $\bn{E}$ and the material logarithmic strain tensor $\bn{E}_H = \ln (\bn{U})$ for the geometric center $\bm{X}_c$ of the HGC\footnote{Note that the log of a tensor is defined by writing it spectrally and replacing each eigenvalue with the log of that eigenvalue. For a case of no shear/off-diagonal terms, you can just take the log of each element on the diagonal to get $\ln(\bn{U})$.}. 
What are the maximum and minimum values of the strain eigenvalues $E_i(t)$ and $E_i^H(t)$? 
Would you expect one set to be more symmetric about zero as $\alpha$ gets large, and why?

\medskip
(d) Determine both the material point acceleration $\bm{A}(\bm{X}_1)$ at \textcolor{red}{\sout{, and spatial acceleration $\bm{a}(\bm{x}_1)$ of material moving through,}} a point $\frac{1}{2} \bm{e}_1 + 2\bm{e}_2 + \frac{1}{2} \bm{e}_3$.  

\begin{figure}
\centering
\animategraphics[loop,autoplay,width=4in]{10}{instr-figures/The_Happy_Gelatinous_Cube-}{1}{10}
\end{figure}

\subsection*{\textbf{1-4 Solution:}}

(a) We are told that the bottom face of the HGC, $X_2=0$, does not move. We will assume that the top face of the HGC, $X_2=2$, only moves in the $\bm{e}_2$ direction and will assure at $X_2=0$ and $X_2=2$, $u_1=u_3=0$. We also assume that while each face will be moving sinusoidally, movement in the $\bm{e}_2$ direction will be opposite in sign  to the sinusoidal movement in the $\bm{e}_1$ and $\bm{e}_3$ directions, so as to ensure that when the cube is in maximum $\bm{e}_2$ compression, the centers of the $\bm{e}_1$ and $\bm{e}_3$ surfaces reach maximum displacements.. With all this, we compose a displacement vector of
\begin{equation*}
    \bm{u}(\bm{X},t)=\begin{bmatrix}
        (2-X_2)X_2X_1\beta\text{sin}\omega t\\ -\frac{\alpha}{2}X_2\text{sin}\omega t \\(2-X_2)X_2X_3\beta\text{sin}\omega t
    \end{bmatrix}.
\end{equation*}
Now, finding $[\bn{F}(\bm{X})]^{\bm{e}}=\gradX\bm{u}+\bn{I}$, we get
\begin{equation*}
    [\bn{F}(\bm{X})]^{\bm{e}}=\begin{bmatrix}
        (2-X_2)X_2\beta\text{sin}\omega t& (2-2X_2)X_1\beta\text{sin}\omega t& 0\\ 0& -\frac{\alpha}{2}\text{sin}\omega t&0 \\0 & (2-2X_2)X_3\beta\text{sin}\omega t&(2-X_2)X_2\beta\text{sin}\omega t 
    \end{bmatrix}+\begin{bmatrix} 1& 0& 0\\ 0& 1& 0\\ 0& 0& 1\end{bmatrix}=
\end{equation*}
\begin{equation*}
    [\bn{F}(\bm{X})]^{\bm{e}}=\begin{bmatrix}
        (2-X_2)X_2\beta\text{sin}\omega t+1& (2-2X_2)X_1\beta\text{sin}\omega t& 0\\ 0& 1-\frac{\alpha}{2}\text{sin}\omega t&0 \\0 & (2-2X_2)X_3\beta\text{sin}\omega t&(2-X_2)X_2\beta\text{sin}\omega t+1 
    \end{bmatrix}
\end{equation*}

\medskip
(b) At $X_2=1$, $[\bn{F}(\bm{X})]^{\bm{e}}$ becomes
\begin{equation*}
    [\bn{F}(\bm{X})]^{\bm{e}}_{X_2=1}=\begin{bmatrix}
        \beta\text{sin}\omega t+1& 0& 0\\ 0& 1-\frac{\alpha}{2}\text{sin}\omega t&0 \\0 & 0&\beta\text{sin}\omega t+1 \end{bmatrix}
\end{equation*}
To find the stretch magnitude of a small fiber at $X_2=1$ and oriented at angle $\theta$ from the $\bm{e}_1$ axis in the $\bm{e}_1-\bm{e}_3$ plane, we want to find $\lambda=\sqrt{\bm{n}\cdot\bn{C}\bm{n}}$ where $\bm{n}$ is the fiber direction and $\bn{C}=\bn{F}^\intercal\bn{F}$. Since at $X_2=1$ $[\bn{F}(\bm{X})]^{\bm{e}}$ is diagonal, it is also symmetric and thus $\bn{F}(\bm{X})=\bn{F}^{\intercal}(\bm{X})$. So,
\begin{gather*}
    \lambda=\sqrt{\bm{n}\cdot\bn{C}\bm{n}}=\sqrt{\bm{n}\cdot\bn{F}^{\intercal}\bn{F}\bm{n}}=\sqrt{\bm{n}\cdot\bn{F}^2\bm{n}}\\
    ([\bn{F}(\bm{X})]^{\bm{e}}_{X_2=1})^2=\begin{bmatrix}
        (\beta\text{sin}\omega t+1)^2& 0& 0\\ 0& (1-\frac{\alpha}{2}\text{sin}\omega t)^2&0 \\0 & 0&(\beta\text{sin}\omega t+1)^2 \end{bmatrix}\\
        \bn{F}^2\bm{n}=\begin{bmatrix}
        (\beta\text{sin}\omega t+1)^2& 0& 0\\ 0& (1-\frac{\alpha}{2}\text{sin}\omega t)^2&0 \\0 & 0&(\beta\text{sin}\omega t+1)^2 \end{bmatrix}\begin{bmatrix}
            \text{cos}\theta\\0\\\text{sin}\theta
        \end{bmatrix}=\begin{bmatrix}
            \text{cos}\theta(\beta\text{sin}\omega t+1)^2\\0\\\text{sin}\theta(\beta\text{sin}\omega t+1)^2
        \end{bmatrix}\\
        \sqrt{\bm{n}\cdot\bn{F}^2\bm{n}}=\sqrt{(\text{cos}^2\theta+\text{sin}^2\theta)(\beta\text{sin}\omega t +1)^2}=\beta\text{sin}\omega t +1
\end{gather*}
So, the stretch magnitude of a small fiber at $X_2=1$ and oriented at angle $\theta$ from the $\bm{e}_1$ axis in the $\bm{e}_1-\bm{e}_3$ plane is 
\begin{equation*}
    \lambda=\beta\text{sin}\omega t +1
\end{equation*}

\medskip
(c) At the geometric center of the cube,
\begin{equation*}
    [\bn{F}(\bm{X}_{c})]^{\bm{e}}=\begin{bmatrix}
        \beta\text{sin}\omega t+1& 0& 0\\ 0& 1-\frac{\alpha}{2}\text{sin}\omega t&0 \\0 & 0&\beta\text{sin}\omega t+1 \end{bmatrix}.
\end{equation*}
At this center, the deformation tensor is again symmetric and diagonal, so, we get
\begin{gather*}
    \bn{E}=\frac{1}{2}(\bn{C}-\bn{I})=\frac{1}{2}(\bn{F}^\intercal\bn{F}-\bn{I})=\frac{1}{2}(\bn{F}^2-\bn{I})\\
    \bn{E}=\frac{1}{2}\begin{bmatrix}
        (\beta\text{sin}\omega t+1)^2-1& 0& 0\\ 0& (1-\frac{\alpha}{2}\text{sin}\omega t)^2-1&0 \\0 & 0&(\beta\text{sin}\omega t+1)^2-1
    \end{bmatrix}
\end{gather*}
Since this strain tensor is already diagonal, at the center of the cube the eigenvectors of the Lagrange-Green strain tensor are the same as our directions $\bm{e}_1$,$\bm{e}_2$, and $\bm{e}_3$. The eigenvalues will then be the three diagonal components of this tensor. A simple analysis shows us that 
\begin{align*}
    E_{min}^{(1)}&=\frac{1}{2}(\beta^2-2\beta)\\
    E_{max}^{(1)}&=\frac{1}{2}(\beta^2+2\beta)\\
    E_{min}^{(2)}&=\frac{1}{2}(\frac{\alpha^2}{4}-\alpha)\\ 
    E_{max}^{(2)}&=\frac{1}{2}(\frac{\alpha^2}{4}+\alpha)\\
    E_{min}^{(3)}&=\frac{1}{2}(\beta^2-2\beta)\\
    E_{max}^{(3)}&=\frac{1}{2}(\beta^2+2\beta)\\
\end{align*}

We find the material logarithmic strain tensor and its eigenvalues at the geometric center in a similar manner.
\begin{gather*}
    \bn{E}_H=\text{ln}(\bn{U})=\text{ln}(\bn{C}^{\frac{1}{2}})=\text{ln}(\bn{F})\\
    \text{ln}(\bn{F})=\begin{bmatrix}
        \text{ln}(\beta\text{sin}\omega t+1)& 0& 0\\ 0& \text{ln}(1-\frac{\alpha}{2}\text{sin}\omega t)&0 \\0 & 0&\text{ln}(\beta\text{sin}\omega t+1)
    \end{bmatrix}
\end{gather*}
Similarly, we get minimum and maximum eigenvalues as
\begin{align*}
    E_{min}^{H(1)}&=\text{ln}(1-\beta)\\
    E_{max}^{H(1)}&=\text{ln}(1+\beta)\\
    E_{min}^{H(2)}&=\text{ln}(1-\frac{\alpha}{2})\\ 
    E_{max}^{H(2)}&=\text{ln}(1+\frac{\alpha}{2})\\
    E_{min}^{H(3)}&=\text{ln}(1-\beta)\\
    E_{max}^{H(3)}&=\text{ln}(1+\beta)\\
\end{align*}
As $\alpha$ gets large, the minimum and maximum of $E^{(2)}$ approach the same value. Thus, we'd expect the set of Lagrange-Green strain eigenvalues to be more symmetric about zero. In contrast, the logarithmic strain eigenvalues, specifically, $E^{H(2)}$, has a minimum that approaches minus infinity as $\alpha$ approaches 2, while the maximum grows more slowly. Thus, I am led to believe this strain will look less symmetric about zero as $\alpha$ gets larger.

\medskip
(d) 
\begin{gather*}
    \bm{u}=\bm{x}-\bm{X}\\
    \bm{x}=\bm{u}+\bm{X}=\begin{bmatrix}
        (2-X_2)X_2X_1\beta\text{sin}\omega t+X_1\\-\frac{\alpha}{2}X_2\text{sin}\omega t+ X_2\\(2-X_2)X_2X_3\beta\text{sin}\omega t+X_3
    \end{bmatrix}\\
    \bm{A}=\frac{\partial \bm{V}}{\partial t}=\frac{\partial^2 \bm{x}}{\partial t^2}=\begin{bmatrix}
        (X_2-2)X_2X_1\beta\omega^2\text{sin}\omega t\\ \frac{1}{2}X_2\alpha\omega^2\text{sin}\omega t\\ (X_2-2)X_2X_3\beta\omega^2\text{sin}\omega t
    \end{bmatrix}\\
    \bm{A}(\bm{X_1})=\begin{bmatrix}
        0\\ \alpha\omega^2\text{sin}\omega t\\ 0
    \end{bmatrix}
\end{gather*}
Note this result aligns with our assumption that at $X_2=2$ the face of the cube only moves in the $\bm{e}_2$ direction.
% This is a placeholder for the example problems from the first problem set. 
% You'll replace this file with the one I supply on canvas. 


%% For citations use: 
%%       \citet{<label>} ==> Lamport (1994)
%%       \citep{<label>} ==> (Lamport, 1994)
%%
%Example citation, See \citet{lamport94}.

%% If you have bib database file and want bibtex to generate the
%% bibitems, please use
%%
%%  \bibliographystyle{elsarticle-harv} 
%%  \bibliography{<your bibdatabase>}

%% else use the following coding to input the bibitems directly in the
%% TeX file.

%% Refer following link for more details about bibliography and citations.
%% https://en.wikibooks.org/wiki/LaTeX/Bibliography_Management

%\begin{thebibliography}{00}

%% For authoryear reference style
%% \bibitem[Author(year)]{label}
%% Text of bibliographic item

\bibliographystyle{elsarticle-harv} 
\bibliography{cas-refs}

%\end{thebibliography}
\end{document}

\endinput
%%
%% End of file `elsarticle-template-harv.tex'.



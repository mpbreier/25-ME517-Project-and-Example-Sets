\setcounter{section}{4} % This causes the next section to be Appendix B


\section*{Examples IV. 3D Time-dependent Behavior Examples}
\label{PS4}

This set of example problems is due on November 17, 2025. 

\medskip
\subsection*{4--1. \textbf{Characterization with deflection history} [4 pts].} 
A simply supported beam \textcolor{red}{with a uniform applied load (or you could instead use a point end load on a cantilevered beam as I mentioned on Slack, either is completely fine)} is made from a material that is well-described by a three-parameter standard linear solid model, i.e., the general creep compliance function $J_c(t)$ is given by
\begin{equation*}
    J_c(t) = J_\infty + (J_0 - J_\infty) \exp\left[-\frac{t}{\tau_c}\right],
\end{equation*}
where $J_\infty$, $J_0$, and $\tau_c$ are all mechanical properties of the material. 

The beam has a length of 4 ft and a second moment of area about the out-of-plane axis of 1 in$^4$. 
It is subjected to a position-time separable load function $q(z,t) = \hat{q}(z) \phi(t)$ where $\phi(t) = -3$ lb/ft $\cdot \mathcal{H}(t)$. 
Say we determine the maximum deflection in the beam at different times to be
\begin{align}
    &u_y \Big|_{\max} = -0.6 \textrm{~in~~at~~}t=30\textrm{~min}\\
    &u_y \Big|_{\max} = -0.75 \textrm{~in~at~~}t=60\textrm{~min}\\
    &u_y \Big|_{\max} = -1.0 \textrm{~in~~at~~}\textrm{~long times}
    \end{align}
From this, calibrate the material properties.

\subsection*{\textbf{4-1 Solution:}}
A beam with a uniform applied load has the following displacement solution in elasticity:
\begin{gather*}
    u_y(z)=\frac{z^4-2Lz^3+L^3z}{24}\cdot\frac{1}{EI_x}\phi_0
\end{gather*}
where the displacement shape function is
\begin{align*}
    \hat{u}_y(z)=\frac{z^4-2Lz^3+L^3z}{24}
\end{align*}.
Through correspondence, we can the express our displacement in the viscoelastic case as
\begin{gather*}
    \bar{u}_y(z,s)=\frac{\hat{u}_y(z)}{I_x}\bar{\phi}(s)\cdot s \bar{J}_c(s)\\
    \bar{u}_y(z,s)=\frac{\hat{u}_y(z)}{(\frac{1}{12}\text{ft})^4}\cdot(-3\frac{\text{lb}}{\text{ft}})\cdot\frac{1}{s}\cdot s \bar{J}_c(s)\\
    \bar{u}_y(z,s)=\frac{\hat{u}_y(z)}{(\frac{1}{12}\text{ft})^4}\cdot(-3\frac{\text{lb}}{\text{ft}})\bar{J}_c(s)=-62,208\frac{\text{lb}}{\text{ft}^5}\cdot\hat{u}(z)\bar{J}_c(s)\\
    u_y(z,t)=-62,208\frac{\text{lb}}{\text{ft}^5}\cdot\hat{u}(z)J_c(t).
\end{gather*}
Because the beam is symmetric, we know the beam will experience max displacement at the center where $z=\frac{1}{2}(4\text{ft})=2\text{ft}$. Using this value of $z$ in the shape function and plugging in the creep compliance function given, we obtain
\begin{gather*}
    u_{y,max}(t)=-207,360\frac{\text{lb}}{\text{ft}^4}(J_\infty + (J_0 - J_\infty) \exp\left[-\frac{t}{\tau_c}\right]) \quad [\text{ft}] 
\end{gather*}
We now can use the data recorded in the problem to create a system of equations and solve for our material properties. Doing so via MATLAB, we find out calibrated material properties as:
\begin{align*}
    &J_0=1.45\cdot 10^{-7}\quad\frac{\text{ft}^5}{\text{lb}}\\
    &J_{\inf}=4.02\cdot10^{-7}\quad\frac{\text{ft}^5}{\text{lb}}\\
    &\tau_c=63.83\quad\text{min}
\end{align*}

\bigskip
\subsection*{4--2. \textbf{Ramp up the torque} [4 pts].} 
A viscoelastic cylinder $AB$ is fixed on its end $A$ and simply supported on its opposite end at $B$. 
At point $B$, a second, rigid rod is joined to the end such that exerting a force on the rod will induce a counterclockwise torque about the axis of $AB$. 
A point force of magnitude $P_0$ is applied to the rigid rod, causing the front face to turn counterclockwise. 
The point force travels along the rod such that its distance from the bar's neutral axis $a(t)$ is direct in time, i.e., $a(t) = \alpha t$, where $\alpha$ is a constant. 
What is the angular rotation of the end $B$ as a function of time, $\Phi(t)$?

\subsection*{\textbf{4-2 Solution:}}
First, we know the torque as a function of time will be linear such that
\begin{gather*}
    T_z=P_0\alpha t
\end{gather*}
Also, for this rod undergoing this stress, the second polar moment of area will be
\begin{gather*}
    J_z=\frac{\pi}{2}R^4
\end{gather*}
where $R$ is the radius of the cylinder. In class, we saw for a bar undergoing such a force, the derivative of the angular rotation can be expressed in the Laplace domain as
\begin{align*}
    \frac{d\bar{\Phi}(z,s)}{dz}&=\frac{1}{J_z}\bar{T}_z(z,s)\cdot s \bar{\mu}_c^{-1}(s)\\
    &=\frac{2}{\pi R^4}\frac{P_0 \alpha}{s^2}\cdot s\bar{\mu}_c^{-1}(s)\\
    &=\frac{2P_0\alpha}{\pi R^4}\frac{1}{s}\bar{\mu}_c^{-1}(s)
\end{align*}
We can then perform an inverse Laplace transform, then integrate this with respect to $z=L$ to obtain $\Phi (L,t)$.
\begin{align*}
    \frac{d\Phi(z,t)}{dz}=\frac{2P_0\alpha}{\pi R^4}\int_0^t\mu_c^{-1}(\tau)d\tau\\
    \Phi(z=L,t)=\int_0^L(\frac{2P_0\alpha}{\pi R^4}\int_0^t\mu_c^{-1}(\tau)d\tau)dz\\
    \Phi(z=L,t)=\frac{2P_0\alpha L}{\pi R^4}\int_0^t\mu_c^{-1}(\tau)d\tau
\end{align*}

This is the solution for the angular rotation of the bar at its end B. One could take this solution further for a given material model such as a Kelvin-Voigt or SLS model.


\bigskip
\subsection*{4--3. \textbf{L-shaped beam} [4 pts].}
\begin{wrapfigure}[8]{r}{2in}
\vspace{-1cm}
    \centering
     \includegraphics[scale=1]{instr-figures/L-bracket.pdf}
    \caption*{Figure 4.3a.}
    \label{fig:Lbracket}
\end{wrapfigure}
A one-piece polymeric member \textit{ABC} consists of two segments of length $L$ at a right angle from one-another and encastred at $A$, as shown. 
Its neutral axes both lie in the $x-z$ plane when unloaded. 
Assuming the creep compliance in shear to be $\mu_c(t)$ and the creep compliance in axial tension to be $E_c(t)$, determine (a) the deflection at $C$ in terms of the time-varying force $F(t)$ and (b) the stress tensor $\mathbf{\sigma}(t)$ at the square surface element located above $A$.  

\subsection*{\textbf{4-3 Solution:}}
(a) We know that from this point load force $F(t)$ there will be torsional deformation in segment \textit{AB} and bending deformation in both segments \textit{AB} and \textit{BC}. If we assume the deflection from torsion as $u_y^{torsion}(z,t)=R \Phi(t)$, we have
\begin{align*}
    u_y^{torsional}(z,t)&=R\frac{2z}{\pi R^4}F(t)*\mu_c^{-1}(t)\\
    &=\frac{2z}{\pi R^3}F(t)*\mu_c^{-1}(t).
\end{align*}
We can find the deflection caused by bending via
\begin{gather*}
    u_y^{bending}(z,t)=\frac{\hat{u}_y(z)}{I_x}\phi(t)*E_c^{-1}(t)
\end{gather*}
where for the given force, we have the deflection shape profile
\begin{gather*}
    \hat{u}_y(z)=\frac{z^3-3Lz^2}{6}
\end{gather*}
and $\phi(t)=F(t)$. If we assume segments \textit{AB} and \textit{BC} will deflect similarly in \textit{y}, we obtain the following:
\begin{align*}
    u_y^{bending}(z,t)&=2\frac{z^3-3Lz^2}{6I_x}F(t)*E_c^{-1}(t)\\
    &=\frac{z^3-3Lz^2}{3I_x}F(t)*E_c^{-1}(t)\quad ; \\
    u_y(z,t)&=u_y^{bending}(z,t)+u_y^{torsional}(z,t)\\
    &=\frac{z^3-3Lz^2}{3I_x}F(t)*E_c^{-1}(t)+\frac{2z}{\pi R^3}F(t)*\mu_c^{-1}(t).
\end{align*}
At \textit{C}, we then have 
\begin{gather*}
    u_y(z=L,t)=\frac{2z^3}{3I_x}F(t)*E_c^{-1}(t)+\frac{2z}{\pi R^3}F(t)*\mu_c^{-1}(t).
\end{gather*}
(b) We will find the stress tensor $\sigma(t)$ by finding the components of the stress tensor separately, specifically we must find the axial stress component and shear stress components. At the surface element at \textit{A}, the moment reaction will simply be the force $F(t)$ multiplied by the length of \textit{BC}. So, we can find the axial stress in Laplace transform as 
\begin{align*}
    \sigma_{zz}(t)&=\frac{M_x(t)R}{I_x}\\
    &=\frac{4F(t)LR}{\pi R^4}\\
    &=\frac{4F(t)L}{\pi R^3}
\end{align*}

We can now find the shear stress component $\sigma_{z\theta}(t)$ at the surface element as
\begin{align*}
    \bar{\sigma}_{z\theta}(s)&=Rs\bar{\mu}_r(s)\cdot \frac{2}{\pi R^4}\bar{F}(s)\cdot\bar{\mu}_c^{-1}(s)\\
    &=\frac{2}{\pi R^3}\frac{1}{s}\bar{F}(s)\\
    \sigma_{z\theta}(t)&=\frac{2}{\pi R^3}\int_0^tF(\tau)d\tau.
\end{align*}
Assembling our stress tensor at the surface element, we obtain:
\begin{gather*}
    \bm{\sigma}(t)=\begin{bmatrix}
        0 & 0 & 0 \\
        0 & 0 & \frac{2}{\pi R^3}\int_0^tF(\tau)d\tau \\
        0 & \frac{2}{\pi R^3}\int_0^tF(\tau)d\tau & \frac{4F(t)L}{\pi R^3}
    \end{bmatrix}
\end{gather*}

\newpage
\subsection*{4--4. \textbf{Fibrous material} [4 pts].}
An elastic material with a fiber phase has the Helmholtz free energy function
\begin{equation*}
\widetilde{\psi}(\bn{F}) = \frac{1}{2}C_{10} (I_1-3) + \frac{1}{4} k (I_a - 1)^2,
\end{equation*}
where $I_1(\bn{C}) = \textrm{tr}(\bn{C})$ and $I_a(\bn{C},\hat{\bm{a}}_0) = \hat{\bm{a}}_0 \cdot \bn{C} \hat{\bm{a}}_0$ for $\bn{C} = \bn{F}^\intercal \bn{F}$.

%\skiponeline
%(a) Show that, given the direction $\hat{\bm{a}}_0$ is fixed by the material, $I_a$ is an invariant of $\bn{C}$.

\medskip
(a) Show that this elastic potential is consistent with the principle of objectivity.

\medskip
(b) Show that the material symmetry group $\mathcal{G}$ is the set of all proper orthogonal tensors $\bn{H}$ with $\bn{H}\hat{\bm{a}}_0 = \hat{\bm{a}}_0$.  

\medskip
(c) Find the first Piola-Kirchhoff and Cauchy stress expressions with an applied deformation gradient tensor of $\bn{F}$.

\subsection*{\textbf{4-4 Solution:}}
(a) For this elastic potential to be consistent with the principle of objectivity, we need to ensure $\widetilde{\psi}(\bm{F}^*)=\widetilde{\psi}(\bm{QF})=\widetilde{\psi}(\bm{F})$ where $\bm{Q}\in\bm{\theta}^+$.
\begin{align*}
    &\bm{C}=\bm{F}^\intercal\bm{F}\\
    &\bm{C}^*=\bm{F}^\intercal\bm{Q}^\intercal\bm{QF} = \bm{F}^\intercal\bm{F} = \bm{C}
\end{align*}
So, it follows then that
\begin{align*}
    &I_1(\bm{C}^*)=I_1(\bm{C})\\
    &I_a(\bm{C}^*,\hat{\bm{a}}_0)=I_a(\bm{C},\hat{\bm{a}}_0)
\end{align*}
and this model thusly is consistent with the principle of objectivity.

\medskip
(b) We need to show the conditions necessary for $\widetilde{\psi}(\bm{FH})=\widetilde{\psi}(\bm{F})$.
\begin{align*}
    &\bm{C}^*=\bm{H}^\intercal\bm{F}^\intercal\bm{F}\bm{H}\\
    &\bm{C}^*=\bm{H}^\intercal\bm{C}\bm{H}\\
    &I_1(\bm{C}^*)=\text{tr}(\bm{H}^\intercal\bm{C}\bm{H})\\
    &I_1(\bm{C}^*)=\text{tr}(\bm{C}\bm{H}\bm{H}^\intercal)\\
    &I_1(\bm{C}^*)=\text{tr}(\bm{C})=I_1(\bm{C})\\\\
    &I_a(\bm{C}^*)=\hat{\bm{a}}_0\bm{H}^\intercal\cdot\bm{C}\bm{H}\hat{\bm{a}}_0
\end{align*}
$I_a(\bm{C}^*)=I_a(\bm{C})$ only if $\bm{H}\hat{\bm{a}}_0=\hat{\bm{a}}_0$. Thus, we have shown the material symmetry group $\mathcal{G}$ must be as stated in the problem statement.

\medskip
(c) We know the first Piola-Kirchhoff stress $\bm{P}=\frac{\partial\widetilde{\psi}}{\partial\bm{F}}$. From the First Piola-Kirchhoff stress we can then find the Cauchy stress via $\bm{\sigma}=J^{-1}\bm{P}\bm{F}^\intercal$.
\begin{align*}
    \bm{P}=\frac{\partial\widetilde{\psi}}{\partial\bm{F}}=\frac{\partial\widetilde{\psi}}{\partial I_1}\frac{\partial I_1}{\partial\bm{F}}+\frac{\partial\widetilde{\psi}}{\partial I_a}\frac{\partial I_a}{\partial\bm{F}}\\
    \frac{\partial\widetilde{\psi}}{\partial I_1}=\frac{1}{2}C_{10}\\
    \frac{\partial\widetilde{\psi}}{\partial I_a}=\frac{1}{2}kI_a\\
    I_1=\text{tr}(\bm{C})=F_{ij}^\intercal F_{ji}=F_{ji}F_{ji}\\
    \frac{\partial I_1}{\partial F_{mn}}=\frac{\partial(F_{ji}F_{ji})}{\partial F_{mn}}=2\delta_{jm}\delta_{ni}F_{ji}=2F_{mn}\\
    \frac{\partial I_1}{\partial\bm{F}}=2\bm{F}\\
    I_a=\hat{\bm{a}}_0 \cdot \bm{C} \hat{\bm{a}}_0=\hat{\bm{a}}_0\bm{F}^\intercal \cdot \bm{F} \hat{\bm{a}}_0=|\bm{F}\hat{\bm{a}}_0|^2\\
    I_a = F_{ij}\hat{a}_{0j}F_{ik}\hat{a}_{0k}\\
    \frac{\partial I_a}{\partial F_{mn}}=\delta_{im}\delta_{jn}\hat{a}_{0j}F_{ik}\hat{a}_{0k}+\delta_{im}\delta_{kn}F_{ij}\hat{a}_{0j}\hat{a}_{0k}=2F_{mk}\hat{a}_k\hat{a}_n\\
    \frac{\partial I_a}{\partial\bm{F}}=2\bm{F\hat{a}}_0\otimes\bm{\hat{a}}_0\\
    \bm{P}=C_{10}\bm{F}+k(I_a-1)\bm{F\hat{a}}_0\otimes\bm{\hat{a}}_0
\end{align*}
Now, we use the above relation to find $\bm{\sigma}$.
\begin{align*}
    \bm{\sigma}=J^{-1}C_{10}\bm{B}+J^{-1}k(I_a-1)(\bm{F}\bm{\hat{a}}_0\otimes\bm{\hat{a}}_0)\bm{F}^\intercal
\end{align*}

\bigskip
\bigskip
\subsection*{4--5. \textbf{Gent model of rubber elasticity} [4 pts].}
The Gent model, which incorporates the finite extensibility of polymer chains in a rubber network but includes $I_2$-dependence, has a\textcolor{red}{n isochoric} free-energy function of $$\bar{\Psi}_{\textcolor{red}{\textrm{iso}}} = -\frac{C_{10}}{2} J_m \ln \left(1 - \frac{I_1-3}{J_m} \right) + C_{01}\ln \left(\frac{I_2}{3}\right), $$ where $C_{10}>0$, $C_{01}\geq 0$, and $J_m>0$ are material constants.

\medskip
(a) Show that the Cauchy stress corresponding to the free energy above has the form\footnote{\textcolor{red}{You may want to use the Cayley-Hamilton theorem to do a replacement for $\bn{B}^2$, which produces $\bn{B}^2 = I_1 \bn{B} - I_2 \bn{I} + I_3 \bn{B}^{-1}$.}} $$\bm{\sigma} = -p \bn{I} + \frac{C_{10} J_m}{J_m - (I_1-3)}\bn{B} - \frac{2C_{01}}{I_2} \bn{B}^{-1}.$$

\medskip
(b) This material undergoes a simple shear deformation of magnitude $\gamma$. Given that the shear modulus $\mu(\gamma^2) \equiv \beta_1(\gamma^2) - \beta_{-1} (\gamma^2)$ where $\beta_i$ corresponds to the expression for $\bm{\sigma}$ in Q1, show that in the small strain limit that $$\lim\limits_{\gamma \rightarrow 0} \mu(\gamma^2) = C_{10} + \frac{2}{3} C_{01}.$$

\subsection*{\textbf{4-5 Solution:}}
(a) As this free-energy function is dependent on the first and second invariants, we can obtain the Cauchy stress via $\bm{\sigma}=J^{-1}\bm{P}\bm{F}^\intercal$ where $J^{-1}=1$ because we are working with an isochoric free-energy function and $\bm{P}$ is given by
\begin{align*}
    \bm{P}= 2(\frac{\partial\psi}{\partial I_1}+ I_1\frac{\partial \psi}{\partial I_2})\bm{F}-2\frac{\partial\psi}{\partial I_2}\bm{BF}+2I_3\frac{\partial \psi}{\partial I_3}\bm{F}^{-\intercal}.
\end{align*}
We can simplify this due to there being no $I_3$ dependence, we simplify to obtain $\bm{\sigma}$ as
\begin{align*}
    \bm{\sigma}=2(\frac{\partial\psi}{\partial I_1}+ I_1\frac{\partial \psi}{\partial I_2})\bm{B}-2\frac{\partial \psi}{\partial I_2}(\bm{B}^2).
\end{align*}
To get rid of $\bm{B}^2$, we use the Cayley-Hamilton theorem to obtain
\begin{align*}
    \bm{\sigma}=2(\frac{\partial\psi}{\partial I_1}+ I_1\frac{\partial \psi}{\partial I_2})\bm{B}-2\frac{\partial \psi}{\partial I_2}(I_1\bm{B}-I_2\bm{I}+I_3\bm{B}^{-1})
\end{align*}
As we are working with an isochoric free-energy density, $I_3=\text{det}\bm{B}=1$. We can then simplify further to obtain
\begin{align*}
    \bm{\sigma}=2 I_2 \frac{\partial\psi}{\partial I_2}\bm{I}+2\frac{\partial\psi}{\partial I_1}\bm{B}-2\frac{\partial\psi}{\partial I_2}\bm{B}^{-1}.
\end{align*}
We now can find the partial derivatives of $\bar{\Psi}_{iso}$ with respect to $I_1$ and $I_2$ as
\begin{align*}
    &\frac{\partial\psi}{\partial I_1}=-\frac{C_{10}}{2}J_m\frac{J_m}{J_m-(I_1-3)}(-\frac{1}{J_m})=\frac{C_{10}}{2}\frac{J_m}{J_m-(I_1-3)}\\
    &\frac{\partial\psi}{\partial I_2}=\frac{C_{01}}{I_2}.
\end{align*}
We plug these into our expression for $\bm{\sigma}$ as
\begin{align*}
    \bm{\sigma}=2C_{01}\bm{I}+\frac{C_{10}J_m}{J_m-(I_1-3)}\bm{B}-\frac{2C_{01}}{I_2}\bm{B}^{-1}.
\end{align*}
As this first term is a hydrostatic pressure term, we can write $2C_{01}=p$, thus yielding the desired form of
\begin{align*}
    \bm{\sigma}=-p\bm{I}+\frac{C_{10}J_m}{J_m-(I_1-3)}\bm{B}-\frac{2C_{01}}{I_2}\bm{B}^{-1}.
\end{align*}

\medskip
(b) First, let us find the corresponding deformation gradient tensor and left Cauchy-Green tensor $\bm{F}$ and $\bm{B}$. For a simple shear deformation,
\begin{align*}
    &\bm{F}=\begin{bmatrix}
        1 & \gamma & 0\\ 0 & 1 & 0\\ 0 & 0 & 1
    \end{bmatrix}\\
    &\bm{B}=\begin{bmatrix}
        1+\gamma^2 & \gamma & 0\\ \gamma & 1 & 0\\ 0 & 0 & 1
    \end{bmatrix}\\
    &\bm{B}^2=\begin{bmatrix}
        (1+\gamma^2)^2+\gamma^2 & \gamma(1+\gamma^2)+\gamma & 0 \\ \gamma(1+\gamma^2)+\gamma & \gamma^2+1 & 0\\ 0 & 0 & 1
    \end{bmatrix}.
\end{align*}
Then,
\begin{align*}
    &I_1=\text{tr}(\bm{B})=3+\gamma^2\\
    &I_2 = \frac{1}{2}((\text{tr}(\bm{B}))^2-\text{tr}(\bm{B}^2))=3+\gamma^2.
\end{align*}
Now, we may write 
\begin{align*}
    \mu(\gamma^2)&= \beta_1(\gamma^2) - \beta_{-1} (\gamma^2)\\
    &=\frac{C_{10}J_m}{J_m-(3+\gamma^2-3)}+\frac{2C_{01}}{3+\gamma^2}.
\end{align*}
It becomes clear that in the small strain limit we do in fact arrive upon
\begin{align*}
    \lim\limits_{\gamma \rightarrow 0} \mu(\gamma^2) &=\frac{C_{10}J_m}{J_m-\gamma^2}+\frac{2C_{01}}{3+\gamma^2}\\
    \lim\limits_{\gamma \rightarrow 0} \mu(\gamma^2) &= C_{10} + \frac{2}{3} C_{01}.
\end{align*}